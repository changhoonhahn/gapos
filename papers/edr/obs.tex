\section{The DESI Bright Galaxy Survey: One-Percent Survey}  \label{sec:edr}
DESI began its five years of operations in May 14, 2021. 
%\todo{something about EDR}
Before its start, DESI conducted the Survey Validation (SV) campagin to verify
that the survey will meets its scientific and performance
requirements. 
The SV campaign was divided into two main programs: the first, SV1,
characterized the survey's performance for different observing conditions and
was used to optimize sample selection. 
The second, the One-Percent Survey (or SV3), observed a dataset that can be
used for representative clustering measurements and deliver a ‘truth’ sample
with high completeness over an area at least 1\% of the expected main survey
footprint.
We refer readers to \cite{sv_paper} for details on the DESI SV programs.
In this work, we focus on BGS galaxies observed during the One-Percent Survey.

The One-Percent Survey observed on 38 nights from April 2021 to the end of 
May 2021.
During this time DESI observed 288 bright time exposures that cover 214 BGS
`tiles', planned DESI pointings. 
The tiles were arranged so that a set of 11 overlapping tiles has their centers 
arranged around a 0.12 deg circle, forming a ‘rosette’ completeness pattern. 
In total, the One-Percent Survey observed 20 rosettes covering 180 
${\rm deg}^2$ spanning the northern galactic cap (see Figure 1 in
\citealt{hahn2022}).  

All BGS spectra observed during the One-Percent Survey are reduced using the
`Fuji' version of the DESI spectroscopic data reduction
pipeline~\citep{guy2022}. 
First, spectra are extracted from the spectrograph CCDs using the 
{\em Spectro-Perfectionsim} algorithm of \cite{bolton2010}.
Then, fiber-to-fiber variations are corrected by flat-fielding and a sky model,
empirically derived from sky fibers, is subtracted from each spectrum.
Afterwards, the fluxes in the spectra are calibrated using stellar model fits
to standard stars. 
The final processed spectra is then derived by co-adding the calibrated spectra
across expoures of the same tile. 
In total, DESI observed spectra of 155,022 BGS Bright and 109,418 BGS Faint 
targets during the One-Percent Survey. 

For each spectrum, redshift is measured using 
{\sc Redrock}\footnote{https://redrock.readthedocs.io}~\citep{bailey2022}, 
a redshift fitting algorithm that uses $\chi^2$ minimization computed from a
linear combination of Principal Component Analysis (PCA) basis spectral
templates in three template classes (``stellar'',  ``galaxy'', and ``quasar'').
{\sc Redrock} also provides measures of redshift uncertainty, $\mathtt{ZERR}$
and redshift confidence, $\Delta\chi^2$, which corresponds to the difference
between the $\chi^2$ values of the best-fit model and the next best-fit model.
We restrict our sample to galaxy targets with reliable redshift measurements.
We only keep targets with spectra classified as galaxy spectra by 
{\sc Redrock}, no {\sc Redrock} warning flags, $\Delta\chi^2 > 40$,
and {\sc Redrock} redshift uncertainty $\mathtt{ZERR} < 0.0005 (1 + z)$.
We also exclude any targets observed using malfunctioning fiber positioners.
Lastly, we impose a redshift range of $0 < z < 0.6$.  
After these cuts, our One-Percent Survey BGS sample includes 143,074 BGS Bright
galaxies and 96,771 BGS Faint galaxies.
