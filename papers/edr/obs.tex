\section{The DESI Bright Galaxy Survey: Early Data Release}  \label{sec:edr}
DESI began its five years of operations in May 14, 2021. 
\todo{something about EDR}
Before its start, DESI conducted the Survey Validation (SV) campagin to verify
that the survey will meets its scientific and performance requirements. 
The SV campaign was divided into two main programs: the first, SV1,
characterized the survey's performance for different observing conditions and
was used to optimize sample selection. 
The second, the One-Percent Survey (or SV3), observed a dataset that can be
used for representative clustering measurements and deliver a ‘truth’ sample
with high completeness over an area at least 1\% of the expected main survey
footprint.
We refer readers to \cite{sv_paper} for details on the DESI SV programs.
In this work, we focus on BGS galaxies observed during the One-Percent Survey.

The One-Percent Survey observed on 38 nights from April 2021 to the end of 
May 2021.
During this time DESI observed 288 bright time exposures that cover 214 BGS
`tiles', planned DESI pointings. 
A set of 11 overlapping tiles so that their centers are arranged around a 0.12
deg circle, forming a ‘rosette’ completeness pattern. 
In total, the One-Percent Survey observed 20 rosettes covering 180 
${\rm deg}^2$ spanning the northern galactic cap (see Figure 1 in
\citealt{hahn2022}).  


In total, DESI observed spectra of \todo{number} BGS Bright and \todo{number}
BGS Faint alaxies during the One-Percent Survey. 
All BGS spectra observed during the One-Percent Survey are reduced using the
`Fuji' version of the DESI spectroscopic data reduction
pipeline~\citep{guy2022}. 
First, spectra are extracted from the spectrograph CCDs using the 
{\em Spectro-Perfectionsim} algorithm of \cite{bolton2010}.
Then, fiber-to-fiber variations are corrected by flat-fielding and a sky model,
empirically derived from sky fibers, is subtracted from each spectrum.
Afterwards, the fluxes in the spectra are calibrated using stellar model fits
to standard stars. 
The final processed spectra is then derived by co-adding the calibrated spectra
across expoures of the same tile. 




%In total, DESI amassed redshifts of 285,335 of BGS Bright galaxies and 201,532 BGS Faint galaxies in ∼5 months of operations. DESI acquired ' 4, 200 redshifts per exposure at a rate of one exposure per 20 minutes on average. Over the next five years, this will expand to an unprecedented >10 million galaxies spanning a third of the sky at two magnitudes deeper than the SDSS MGS. In Figure 18, we highlight the progress of BGS and present the absolute number of redshifts in ∆z = 0.02 bins for BGS Bright (left; blue) and Faint (right; orange) galaxies observed during the SV programs. BGS targeting successfully delivers galaxies spanning the desired redshift range, 0.0 < z < 0.6. The Bright and Faint samples have median redshifts of z = 0.2 and 0.3 respectively, which is more than double that of the SDSS MGS. We include the redshift distribution of GAMA DR4 for comparison. Even with the EDR alone, BGS exceeds the total number of spectroscopic redshift of GAMA.
