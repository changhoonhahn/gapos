\documentclass[12pt, letterpaper, preprint, comicneue]{aastex63}
%\usepackage[default]{comicneue} % comic sans font for editing
\usepackage[T1]{fontenc}
\input{vc}
\usepackage{color}
\usepackage{amsmath}
\usepackage{natbib}
\usepackage{ctable}
\usepackage{bm}
\usepackage[normalem]{ulem} % Added by MS for \sout -> not required for final version
\usepackage{xspace}
\usepackage{csvsimple} 

\usepackage{graphicx}
\usepackage{pgfkeys, pgfsys, pgfcalendar}


% typesetting shih
\linespread{1.08} % close to 10/13 spacing
\setlength{\parindent}{1.08\baselineskip} % Bringhurst
\setlength{\parskip}{0ex}
\let\oldbibliography\thebibliography % killin' me.
\renewcommand{\thebibliography}[1]{%
  \oldbibliography{#1}%
  \setlength{\itemsep}{0pt}%
  \setlength{\parsep}{0pt}%
  \setlength{\parskip}{0pt}%
  \setlength{\bibsep}{0ex}
  \raggedright
}
\setlength{\footnotesep}{0ex} % seriously?

% citation alias

% math shih
\newcommand{\setof}[1]{\left\{{#1}\right\}}
\newcommand{\given}{\,|\,}
\newcommand{\lss}{{\small{LSS}}\xspace}

\newcommand{\Om}{\Omega_{\rm m}} 
\newcommand{\Ob}{\Omega_{\rm b}} 
\newcommand{\OL}{\Omega_\Lambda}
\newcommand{\smnu}{M_\nu}
\newcommand{\sig}{\sigma_8} 
\newcommand{\mmin}{M_{\rm min}}
\newcommand{\BOk}{\widehat{B}_0} 
\newcommand{\hmpc}{\,h/\mathrm{Mpc}}
\newcommand{\bfi}[1]{\textbf{\textit{#1}}}
\newcommand{\parti}[1]{\frac{\partial #1}{\partial \theta_i}}
\newcommand{\partj}[1]{\frac{\partial #1}{\partial \theta_j}}
\newcommand{\mpc}{{\rm Mpc}}
\newcommand{\eg}{\emph{e.g.}}
\newcommand{\ie}{\emph{i.e.}}

% cmds for this paper 
\newcommand{\gr}{g{-}r}
\newcommand{\fnuv}{FUV{-}NUV}
\newcommand{\sfr}{{\rm SFR}}
\newcommand{\ssfr}{{\rm SSFR}}
\newcommand{\mtaum}{m_{\tau,M_*}}
\newcommand{\mtaus}{m_{\tau,{\rm SSFR}}}
\newcommand{\ctau}{c_\tau}
\newcommand{\mdeltam}{m_{\delta,M_*}}
\newcommand{\mdeltas}{m_{\delta,{\rm SFR}}}
\newcommand{\cdelta}{c_\delta}
\newcommand{\eda}{EDA}


\newcommand{\specialcell}[2][c]{%
  \begin{tabular}[#1]{@{}c@{}}#2\end{tabular}}
% text shih
\newcommand{\foreign}[1]{\textsl{#1}}
\newcommand{\etal}{\foreign{et~al.}}
\newcommand{\opcit}{\foreign{Op.~cit.}}
\newcommand{\documentname}{\textsl{Article}}
\newcommand{\equationname}{equation}
\newcommand{\bitem}{\begin{itemize}}
\newcommand{\eitem}{\end{itemize}}
\newcommand{\beq}{\begin{equation}}
\newcommand{\eeq}{\end{equation}}

%% collaborating
\newcommand{\todo}[1]{\marginpar{\color{red}TODO}{\color{red}#1}}
\definecolor{orange}{rgb}{1,0.5,0}
\newcommand{\ch}[1]{{\color{orange}{\bf CH:} #1}}

\begin{document} \sloppy\sloppypar\frenchspacing 

\title{PROVABGS Probabilistic Stellar Mass Function of the BGS One-Percent
Survey} 
\date{\texttt{DRAFT~---~\githash~---~\gitdate~---~NOT READY FOR DISTRIBUTION}}

\newcounter{affilcounter}
\author{ChangHoon Hahn}
\altaffiliation{changhoon.hahn@princeton.edu}
\affil{Department of Astrophysical Sciences, Princeton University, Peyton Hall, Princeton NJ 08544, USA} 
\author{DESI collaboration} 

\begin{abstract}
    We present the probabilistic stellar mass function (pSMF) of galaxies in
    the DESI Bright Galaxy Survey (BGS), observed during the One-Percent
    Survey. 
    The One-Percent Survey was one of DESI's survey validation programs that
    was conducted from April to May 2021, before the start of the main DESI
    survey. 
    It observed with the same target selection and observing strategy as the
    main survey and successfully observed the spectra of 143,017 galaxies in
    the $r < 19.5$ magnitude-limited BGS Bright sample and 95,499 galaxies in
    the fainter surface brightness and color selected BGS Faint sample over 
    $0 < z < 0.6$.
    We derive pSMFs from posteriors of stellar mass, $M_*$, inferred from
    DESI photometry and spectroscopy using the \cite{hahn2022} PROVABGS
    Bayesian SED modeling framework. 
    We use a hierarchical population inference framework that statistically
    rigorously propagates the $M_*$ uncertainties. 
    Furthermore, we include correction weights that account for the selection
    effects and incompleteness of the BGS observations. 
    We present the redshift evolution of the pSMF in BGS as well as the pSMFs
    of star-forming and quiescent galaxies classified using average specific
    star formation rates from PROVABGS. 
    Overall, the pSMFs show good agreement with previous SMF measurements in
    the literature. 
    Our pSMFs showcase the potential and statistical power of BGS, which in its
    main survey will observe >100$\times$ more galaxies.
    Moreover, we present the population inference framework for subsequent
    population statistics measurements using BGS, which will characterize the
    global galaxy population and galaxy scaling relations at low redshifts with
    unprecendent precision. 
\end{abstract}

\keywords{
    cosmology: observations -- galaxies: evolution -- galaxies: statistics
}

% --- intro ---  
\section{Introduction} \label{sec:intro} 
brief review of scaling relations and the importance of population statistics  

list of all the population statistics paper from SDSS and GAMA. 

introduce DESI and BGS and talk about an opportunity to improve all these
population population statistics 

{\color{red} why do we care about smfs} 
importance of pop statistics for galaxy formation models. 
UniverseMachine,

informing semianalytic models and hydrodyanmical simulations

ML acceleration of these models will enable to fitting to observations so it's
imperative that we have the most precise galaxy population statistics. 

{\color{red} why provabgs} 
state of the art SED modeling 
fully consistent way of examining M*, sfr, metallicity, and dust content of
galaxies. 

fully probabilistic for rigorous pSMF measurements. 

{\color{red} why this work?} 
In this work, we present the pSMF for galaxies in the Bright Galaxy Survey
observed as part of the DESI One-Percent Survey, a survey validation program
conducted before the main survey operations. 

Furthermore, in this work we present the statistical methodology as well as the
methodology for accounting for observational incompleteness. 

% --- observations ---  
\section{The DESI Bright Galaxy Survey: One-Percent Survey}  \label{sec:edr}
DESI began its five years of operations in May 14, 2021. 
%\todo{something about EDR}
Before its start, DESI conducted the Survey Validation (SV) campagin to verify
that the survey will meets its scientific and performance
requirements. 
The SV campaign was divided into two main programs: the first, SV1,
characterized the survey's performance for different observing conditions and
was used to optimize sample selection. 
The second, the One-Percent Survey (or SV3), observed a dataset that can be
used for representative clustering measurements and deliver a ‘truth’ sample
with high completeness over an area at least 1\% of the expected main survey
footprint.
We refer readers to \cite{sv_paper} for details on the DESI SV programs.
In this work, we focus on BGS galaxies observed during the One-Percent Survey.

The One-Percent Survey observed on 38 nights from April 2021 to the end of 
May 2021.
During this time DESI observed 288 bright time exposures that cover 214 BGS
`tiles', planned DESI pointings. 
The tiles were arranged so that a set of 11 overlapping tiles has their centers 
arranged around a 0.12 deg circle, forming a ‘rosette’ completeness pattern. 
In total, the One-Percent Survey observed 20 rosettes covering 180 
${\rm deg}^2$ spanning the northern galactic cap (see Figure 1 in
\citealt{hahn2022}).  

All BGS spectra observed during the One-Percent Survey are reduced using the
`Fuji' version of the DESI spectroscopic data reduction
pipeline~\citep{guy2022}. 
First, spectra are extracted from the spectrograph CCDs using the 
{\em Spectro-Perfectionsim} algorithm of \cite{bolton2010}.
Then, fiber-to-fiber variations are corrected by flat-fielding and a sky model,
empirically derived from sky fibers, is subtracted from each spectrum.
Afterwards, the fluxes in the spectra are calibrated using stellar model fits
to standard stars. 
The final processed spectra is then derived by co-adding the calibrated spectra
across expoures of the same tile. 
In total, DESI observed spectra of 155,022 BGS Bright and 109,418 BGS Faint 
targets during the One-Percent Survey. 

For each spectrum, redshift is measured using 
{\sc Redrock}\footnote{https://redrock.readthedocs.io}~\citep{bailey2022}, 
a redshift fitting algorithm that uses $\chi^2$ minimization computed from a
linear combination of Principal Component Analysis (PCA) basis spectral
templates in three template classes (``stellar'',  ``galaxy'', and ``quasar'').
{\sc Redrock} also provides measures of redshift uncertainty, $\mathtt{ZERR}$
and redshift confidence, $\Delta\chi^2$, which corresponds to the difference
between the $\chi^2$ values of the best-fit model and the next best-fit model.
We restrict our sample to galaxy targets with reliable redshift measurements.
We only keep targets with spectra classified as galaxy spectra by 
{\sc Redrock}, no {\sc Redrock} warning flags, $\Delta\chi^2 > 40$,
and {\sc Redrock} redshift uncertainty $\mathtt{ZERR} < 0.0005 (1 + z)$.
We also exclude any targets observed using malfunctioning fiber positioners.
Lastly, we impose a redshift range of $0 < z < 0.6$.  
After these cuts, our One-Percent Survey BGS sample includes 143,074 BGS Bright
galaxies and 96,771 BGS Faint galaxies.

% --- methods ---  
\section{PROVABGS SED Modeling} \label{sec:provabgs}
% brief explanation of the PROVABGS SED modeling 
For each BGS galaxy, we derive its $M_*$ and other properties, 
$\overline{\rm SFR}$, $Z_{\rm MW}$, and $t_{\rm age, MW}$ from DESI
photometry and spectroscopy using the PROVABGS SED modeling
framework~\citep{hahn2022}.  
PROVABGS models galaxy SEDs using stellar population synthesis with
non-parametric star-formation history (SFH) with a starburst, a non-parametric
metallicity history (ZH) that varies with time, and a flexible dust
attenuation prescription.
The non-parameteric SFH and ZH prescriptions are derived from SFHs and ZHs of
simulated galaxies in the Illustris hydrodynamic
simulation~\citep{vogelsberger2014, genel2014, nelson2015} and provide compact 
and flexibly representations of SFHs and ZHs.
For the stellar population synthesis, PROVABGS uses the Flexible Stellar
Population Synthesis~\citep[FSPS;][]{conroy2009, conroy2010b} model with MIST
isochrones~\citep{paxton2011, paxton2013, paxton2015, choi2016, dotter2016},
\cite{chabrier2003} initial mass function (IMF), and a combination of
MILES~\citep{sanchez-blazquez2006} and BaSeL~\citep{lejeune1997, lejeune1998,
westera2002} spectral libraries.

Furthermore, PROVABGS provides a Bayesian inference framework that infers
full posterior probability distributions of the SED model parameter:
$p(\theta\given {\bf X}^{\rm photo}, {\bf X}^{\rm spec})$, where 
${\bf X}^{\rm photo}$ represents the photometry and ${\bf X}^{\rm spec}$ 
represents the spectroscopy. 
In total, $\theta$ has 13 parameters: $M_*$, 6 parameters specifying the SFH
($\beta_1, \beta_2, \beta_3, \beta_4, f_{\rm burst}, t_{\rm burst}$), 2
parameters specifying ZH ($\gamma_1, \gamma_2$), 3 parameters specifying
dust attenuation ($\tau_{\rm BC}, \tau_{\rm ISM}, n_{\rm dust}$), and a
nuisance parameter for the fiber aperture effect. 
Posteriors have distinct advantages over point estimates because they
accurately estimate uncertainties and degeneracies among galaxy properties.
Furthermore, as we later demonstrate, they are essential for principled
population inference: \eg~SMF.  

In practice, accurately estimating a 13 dimensional posterior requires a large
number ($\gtrsim$100,000) SED model evaluations, which requires prohibitive
computational resources --- $\sim 10$ CPU hours per galaxy. 
To address this challenge, PROVABGS samples the posterior using the
\cite{karamanis2020} ensemble slice Markov Chain Monte Carlo (MCMC) sampling
with the {\sc zeus} Python package\footnote{https://zeus-mcmc.readthedocs.io/}.
PROVABGS further accelerates the inference by using neural emulators for the
SED models. 
The emulators are accurate to subpercent level and $>100\times$ faster than the
original SED model based on FSPS~\citep{kwon2022}. 
With {\sc zeus} and neural emulation, deriving a posterior takes $\sim$5 min
per galaxy with PROVABGS.
Moreover, \cite{hahn2022} demonstrated PROVABGS can accurately infer $M_*$
overall the full expected $M_*$ range of BGS, using forward modeled synthetic
DESI observations. 

\begin{figure}
\begin{center}
    \includegraphics[width=0.6\textwidth]{figs/provabgs_posterior.pdf}
    \caption{
        {\em Top panels}: 
        Posteriors of galaxy properties, $M_*$, $\overline{\rm SFR}$, 
        $Z_{\rm MW}$, and $t_{\rm age, MW}$, for a randomly selected BGS 
        Bright galaxy with $z=0.2242$ (target ID: 39627757520424630) inferred
        using the PROVABGS SED modeling framework from DESI photometry and
        spectroscopy. 
        The contours mark the {\color{red} X, X, and X} percentiles of
        posterior. 
        With the PROVABGS posteriors, we accurately estimate the galaxy
        properties, their uncertainties, and any degeneracies among them. 
        {\em Bottom panels}: 
        Comparison of the best-fit PROVABGS SED model prediction (black) to
        observations (blue). 
        We compare the $g$, $r$, and $z$ band photometry in the left panel and 
        spectra in the right panel. 
        We infer the posterior of galaxy properties for every BGS galaxies in
        the DESI One-Percent Survey.
    }\label{fig:posterior}
\end{center}
\end{figure}


\begin{figure}
\begin{center}
    \includegraphics[width=0.6\textwidth]{figs/mstar_z.pdf}
    \caption{
        $M_*$ as a function of $z$ of BGS Bright (blue) and Faint (orange)
        galaxies in the DESI One-Percent Survey. 
        For $M_*$, we use the best-fit values derived using PROVABGS. 
        BGS Bright is a magnitude-limited sample to $r < 19.5$ while BGS Faint
        includes fainter galaxies $19.5 < r < 20.175$ selected using $r_{\rm
        fib}$ and color~\citep{hahn2022a}. 
        In total, we infer the posteriors of 143,017 BGS Bright and 95,499 BGS
        Faint galaxies in the DESI One-Percent Survey spanning $0 < z < 0.6$. 
    }\label{fig:mstar_z}
\end{center}
\end{figure}


In Figure~\ref{fig:posterior}, we demonstrate the PROVABGS SED modeling
framework for a randomly selected BGS Bright galaxy with $z=0.2242$ 
(target ID: 39627757520424630).
In the top panels, we present the posteriors of galaxy properties, $M_*$, 
$\overline{\rm SFR}$, $Z_{\rm MW}$, and $t_{\rm age, MW}$, inferred from DESI
photometry and spectroscopy. 
We mark the {\color{red} X, X, and X} percentiles of posterior with the
contours. 
The posteriors illustrate that we can precisely measure the properties of BGS
galaxies from DESI photometry and spectroscopy. 
Furthermore, with the full posterior, we accurately estimate the uncertainties
on the galaxy properties and the degeneracies among them (\emph{e.g.} $M_*$ and
$\overline{\rm SFR}$). 
In the bottom panels, we compare the PROVABGS SED model prediction using the
best-fit parameter values (black) to DESI observations (blue). 
The left panel compares the optical $g$, $r$, and $z$ band photometry while the
right panel compares the spectra. 
The comparison shows good agreement between the best-fit model and the
observations. 

We derive a PROVABGS posterior (\emph{e.g.} Figure~\ref{fig:posterior}) for
every galaxy in the DESI One-Percent Survey. 
In Figure~\ref{fig:mstar_z}, we present the best-fit $M_*$ measurements as a
function of $z$ for the BGS galaxies in DESI One-Percent Survey. 
We mark the galaxies in the BGS Bright sample in blue and the ones in the BGS
Faint sample in orange. 
We infer the posteriors of 143,017 BGS Bright and 95,499 BGS Faint galaxies. 

% --- results ---  
\section{Results} \label{sec:results}
From the posteriors of galaxy properties inferred using
PROVABGS~(Section~\ref{sec:provabgs}), we derive the marginalized posteriors: 
$p(M_* \given {\bfi X_i})$, the marginalized 1D posterior of $M_*$ from
observed spectrophotometry ${\bfi X_i}$ of galaxy $i$.
Using these posteriors, we can estimate the probabilistic SMF (pSMF) of BGS
galaxies using population inference in a hierarchical Bayesian 
framework~\citep[\emph{e.g.}][]{hogg2010, foreman-mackey2014, baronchelli2020}.
In other words, we can infer $p(\phi\given\{{\bfi X_i}\})$, the probability
distribution of $\phi$ given the full DESI BGS observations, $\{{\bfi X_i}\}$. 
$\phi$ is the set of population hyperparameters that describe the pSMF,
$\Phi(M_*; \phi)$.
We again emphasize that this approach is statistically rigorous and correctly
propagates the uncertainties in our $M_*$ measurements to the pSMF. 

%We take this approach over the standard approach that use point estimates of
%$M_*$ because we can correctly propagate the uncertainties in our $M_*$
%measurements and more robustly estimate the $M_*$ distribution --- \emph{ie.}
%SMF. 
%\cite{malz2020} demonstrated in the context of inferring redshift
%distributions from individual photometric redshift measurements that using
%point estimates is statistically incorrect and can lead to biased redshift
%distributions. 
%We emphasize that \cite{malz2020} is an analogous analysis with a similar goal
%of measuring a 1D galaxy property distribution. 

In this work, we estimate the pSMF using a Gaussian Mixture
Model~\citep[GMM;][]{press1992, mclachlan2000}, which provides a highly flexible
description of the $M_*$ distribution: 
\begin{equation}
    \Phi(M_*; \phi) = \sum\limits_{j=1}^{k} \mathcal{N}(M_*; \phi_j).
\end{equation} 
$k$ represents the number of Gaussian components. 
$\phi_j$ represent the mean and standard deviation of the $j^{\rm th}$ Gaussian
component of the GMM. 
Previous works have used parametric functions (\emph{e.g.} Schechter function)
to describe the pSMF~\citep{leja2019a}. 
We opt for GMMs in order to produce a non-parametric measuerment of the pSMF.
In a subsequent work, Speranza~\etal~(in prep.), we will present the BGS pSMF
measured using a parametric model with continuous redshift evolution.  


To infer $p(\phi\given\{{\bfi X_i}\})$, we follow the same approach described
in \cite{hahn2022}:
\begin{align}\label{eq:popinf}
p(\phi \given \{{\bfi X_i}\}) 
    =&~\frac{p(\phi)~p( \{{\bfi X_i}\} \given \phi)}{p(\{{\bfi X_i}\})}\\
    =&~\frac{p(\phi)}{p(\{{\bfi X_i}\})}\int p(\{{\bfi X_i}\} \given \{\theta_i\})~p(\{\theta_i\} \given \phi)~{\rm d}\{\theta_i\}.\\
    =&~\frac{p(\phi)}{p(\{{\bfi X_i}\})}\prod\limits_{i=1}^N\int p({\bfi X_i} \given \theta_i)~p(\theta_i \given \phi)~{\rm d}\theta_i\\
    =&~\frac{p(\phi)}{p(\{{\bfi X_i}\})}\prod\limits_{i=1}^N\int \frac{p(\theta_i \given {\bfi X_i})~p({\bfi X_i})}{p(\theta_i)}~p(\theta_i \given \phi)~{\rm d}\theta_i\\
    =&~p(\phi)\prod\limits_{i=1}^N\int \frac{p(\theta_i \given {\bfi X_i})~p(\theta_i \given \phi)}{p(\theta_i)}~{\rm d}\theta_i. 
\end{align}
\noindent We can estimate the integral using $S_i$ Monte Carlo samples from
the individual posteriors $p(\theta_i \given {\bfi X_i})$: 
\begin{align}
    p(\phi \given \{{\bfi X_i}\}) \approx~p(\phi)\prod\limits_{i=1}^N\frac{1}{S_i}\sum\limits_{j=1}^{S_i}
    \frac{p(\theta_{i,j} \given \phi)}{p(\theta_{i,j})}.
\end{align} 

%BGS provides two samples: BGS Bright and Faint. 
%Galaxies in BGS Bright are selected based on a $r < 19.5$ flux limit, while
%the ones in BGS Faint are selected based on a fiber-magnitude and color limit
%and $r < 20.0175$ flux limit. 
Since the sample of BGS galaxies is not volume-limited and complete as a
function of $M_*$, we must account for the selection effect and incompleteness
when estimating the pSMF. 
To account for the selection effects of the BGS samples, we include weights
derived from $z^{\rm max}_i$, the maximum redshift that galaxy $i$ could have
and still be included in the BGS samples. 
We derive $z^{\rm max}_i$ for every galaxy using by redshifting the SED
predicted by the best-fit parameters. 
We then derive $V^{\rm max}_i$, the comoving volume out to $z^{\rm max}_i$, and
include a factor of $1/V^{\rm max}_i$ in the galaxy weight $w_i$. 

Next, we include correction weights for spectroscopic incompleteness driven by
fiber assignment and redshift failures. 
Incompletenss from fiber assignment is due to the fact that DESI is not able to
assign fibers to all galaxies included in the BGS target selection. 
Furthermore, due to the clustering of galaxies there is significant variation
in the assignment probability. 
Meanwhile, incompleteness from redshift failure is caused by the fact that we
do not successfully measure the redshift for every spectra and the redshift
failure rate depends significantly on the surface brightnesses of the galaxies
and the signal-to-noise ratio of the spectra. 
We describe how we derive  the incompleteness correction weights for fiber
assignment and redshift failures, $w_{i, {\rm FA}}$ and $w_{i, {\rm ZF}}$, in
Appendix~\ref{sec:spec_comp}. 
Each BGS galaxy is assigned a weight of 
$w_i = (w_{i, {\rm FA}}\times w_{i, {\rm ZF}})/V^{\rm max}_i$.

We modify Eq.~\ref{eq:popinf} to include galaxy weights, $w_i$: 
\begin{align}
p(\phi \given \{{\bfi X_i}\}) 
    \approx&~\frac{p(\phi)}{\prod\limits_{i=1}^N p({\bfi X_i})^{w_i}} 
    \prod\limits_{i=1}^N \left(\int p({\bfi X_i} \given \theta_i)~p(\theta_i \given \phi)~{\rm d}\theta_i \right)^{w_i} \\ 
    \approx&~\frac{p(\phi)}{\prod\limits_{i=1}^N p({\bfi X_i})^{w_i}} 
    \prod\limits_{i=1}^N \left( \sum\limits_{j=1}^{S_i}
    \frac{p(\theta_{i,j} \given \phi)}{p(\theta_{i,j})} \right)^{w_i} \\
    \approx&~\frac{p(\phi)}{\prod\limits_{i=1}^N p({\bfi X_i})^{w_i}} 
    \prod\limits_{i=1}^N \left( \sum\limits_{j=1}^{S_i}
    \frac{q_\phi(\theta_{i,j})}{p(\theta_{i,j})} \right)^{w_i}.
\end{align} 

In practice, we do not derive the full posterior 
$p(\phi \given \{{\bfi X_i}\})$. 
Instead we derive the maximum a posteriori (MAP) hyperparameter 
$\phi_{\rm MAP}$ that maximizes $p(\phi \given \{{\bfi X_i}\})$ or 
$\log p(\phi \given \{{\bfi X_i}\})$.
We expand, 
\begin{align}
\log p(\phi \given \{{\bfi X_i}\}) 
    \approx&~\log p(\phi) + % \sum\limits_{i=1}^N w_i \log w_i + 
    \sum\limits_{i=1}^N w_i \log \left(\sum\limits_{j=1}^{S_i} \frac{q_\phi(\theta_{i,j})}{p(\theta_{i,j})} \right).
\end{align} 
Since the first two terms are constant, we derive $\phi_{\rm MAP}$ by
maximizing 
\begin{equation}
    \max_\phi~~\sum\limits_{i=1}^N w_i \log \left(\sum\limits_{j=1}^{S_i} \frac{q_\phi(\theta_{i,j})}{p(\theta_{i,j})} \right).
\end{equation}
using the {\sc Adam} optimizer~\citep{kingma2017}.  
We derive $\phi_{\rm MAP}$ for BGS galaxies in redshift bins of width 
$\Delta z = 0.04$ starting from $z =0.01$ in order to examine the redshift
evolution of the SMF within BGS. 

%\approx&~\log p(\phi) - 
%\log \prod\limits_{i=1}^N p({\bfi X_i})^{w_i} + 
%\log \prod\limits_{i=1}^N \left(\int p({\bfi X_i} \given \theta_i)~p(\theta_i \given \phi)~{\rm d}\theta_i \right)^{w_i} \\
%\approx&~\log p(\phi) - 
%\sum\limits_{i=1}^N w_i \log p({\bfi X_i}) + 
%\sum\limits_{i=1}^N w_i \log \left(\int p({\bfi X_i} \given \theta_i)~p(\theta_i \given \phi)~{\rm d}\theta_i \right) \\
%\approx&~\log p(\phi) + 
%\sum\limits_{i=1}^N w_i \log \left(\frac{1}{w_i} \sum\limits_{j=1}^{S_i} w_{i,j} \frac{p(\theta_{i,j} \given \phi)}{p(\theta_{i,j}} \right) \\

%\subsection{Targeting Completeness} \label{sec:ts}
% https://desi.lbl.gov/trac/wiki/ClusteringWG/LSScat/SV3/version2.1/fulldat
% https://desi.lbl.gov/trac/wiki/ClusteringWG/LSScat/SV3/version2.1/fullran

\begin{figure}
\begin{center}
    \includegraphics[width=0.9\textwidth]{figs/psmf_bgs_any_comp.pdf} 
    \caption{
        The probabilistic SMF (pSMF) of BGS galaxies in the One-Percent Survey
        at $0.01 < z < 0.05$ (black line). 
        We represent uncertainties on the pSMF, estimated using a standard
        jackknife technique (Appendix~\ref{sec:jack}), in the shaded regions.
        The solid line represents the pSMF above the completeness limit 
        $M_* > M_{\rm lim} = 10^{8.975}M_\odot$ (Appendix~\ref{sec:mscomp}).
        In the left panel, we present the pSMFs of BGS Bright (blue) and
        Faint (orange) galaxies. 
        In the right panel, we include SMF measurements from previous
        spectroscopic surveys for comparison: SDSS~\citep{moustakas2013,
        bernardi2017} and GAMA~\citep{driver2022}. 
        Overall, the pSMF of BGS are in good agreement with SMF
        measurements from previous surveys.  
    }\label{fig:psmf}
\end{center}
\end{figure}

\begin{figure}
\begin{center}
    \includegraphics[width=0.5\textwidth]{figs/psmf_bgs_any_zevo.pdf} 
    \caption{
        The BGS pSMF over the redshift range $0.01 < z < 0.17$ in bins of
        $\Delta z = 0.04$. 
        The shaded regions represent the uncertainties on the pSMF, estimated
        using a standard jackknife technique.
        The solid lines represent the pSMF above the completeness limit 
        $M_* > M_{\rm lim}$ while the dashed lines represent the pSMF below
        the limit.
        There is no significant redshift evolution of the pSMF given the
        statistical uncertainties. 
        The main BGS survey will observe $>100\times$ more galaxies than the
        One-Precent Survey. 
    }\label{fig:psmfz}
\end{center}
\end{figure}

\subsection{The Probabilistic Stellar Mass Function} \label{sec:psmf}
We present the probabilistic SMF (pSMF) of $0.01 < z < 0.05$ BGS galaxies in the
One-Percent Survey in Figure~\ref{fig:psmf} (black line). 
The shaded regions represent the uncertainties of the pSMF from sample variance,
which we derive using a standard jackknife technique
(Appendix~\ref{sec:jack}) and are conservative estimates~\citep{norberg2009}. 
In the left panel, we also present the pSMFs of the BGS Bright (blue) and Faint
(orange) galaxies.  
BGS Bright galaxies are selected using a $r > 19.5$ magnitude limit. 
As a result, the BGS Bright sample is $M_*$ complete above $M_{\rm lim} >
10^{8.975}M_\odot$. 
We derive $M_{\rm lim}$ in Appendix~\ref{sec:mscomp} and mark the pSMF above
the completeness limit in solid and below the limit in dashed. 
Meanwhile, the BGS Faint sample is selected using a surface brightness and
color selection.   
It includes fainter galaxies, $19.5 < r < 20.175$, with overall lower $M_*$
than the BGS Bright sample. 

We also include the SMF estimated using the standard approach (black dotted). 
This SMF is derived using the best-fit $M_*$ point estimates with the same
galaxy weights, $w_i$. 
At intermediate $M_*$ range, $10^9 < M_* < 10^{11}M_\odot$, we find good
agrement with the pSMF. 
However, the standard approach significantly underestimates the SMF outside
this $M_*$ range. 
These discrepancies is due to the fact that point estimates of $M_*$ ignore the
uncertainties which contribute significantly at the most and least massive ends
of the SMF.
The discpreancies are present in all other redshift bins and underscore the
importance of correctly propagating the $M_*$ uncertainties. 

In the right panel, we compare the BGS pSMF to SMF measurements from previous 
spectroscopic surveys: SDSS~\citep{moustakas2013, bernardi2017} (black circle
and square) and GAMA~\citep{driver2022} (black triangle).
%For the \cite{driver2022} SMF, we include a 0.0807 dex correction that the authors recommend to re-normalize the SMF and correction to $z=0$. 
We note that there is significant variance in SMF measurements in the
literature, especially at the high $M_*$ end. 
This is partly due to the different modeling methodologies used to derive
$M_*$, which can contribute >0.1 dex discrepancies~\citep{pacifici2023}. 
Furthermore, there are also discrepancies due to photometric corrections
applied to SDSS photometry, assumptions on the stellar populations, and
dust~\citep{bernardi2017}.
In a subsequent work, we will present a detailed comparison of BGS $M_*$
measurements using different methods. 
Overall, we find good agreement with previous SMF measurements, especially in
the intermediate $M_*$ range where we precisely infer the pSMF.  

In Figure~\ref{fig:psmfz}, we present the redshift evolution of the pSMF over 
$0.01 < z < 0.17$ in redshift bins of $\Delta z = 0.04$. 
The shaded region represent the jackknife uncertainties for the pSMF.
The solid line represents the pSMF above $M_{\rm lim}$ while the dashed lines
represent the pSMF below the limit. 
We only include 4 redshift bins, since $M_{\rm lim} > 10^{10.5}M_\odot$ for 
$z > 0.17$ (Table~\ref{tab:mscomp}).
The pSMFs in Figure~\ref{fig:psmfz} do not reveal a significant redshift
dependence given their uncertainties. 
We note that the large uncertainties for the $0.01 < z < 0.05$ pSMF is driven
by large-scale structure at RA $\sim 195$ deg, Dec $\sim 28$ deg, and 
$z\sim0.244$. 
The main BGS survey will observe $>100\times$ more BGS galaxies than the
One-Percent Survey and enable pSMF measurements with unprecedented precision. 

\begin{figure}
\begin{center}
    \includegraphics[width=0.5\textwidth]{figs/sfq.pdf} 
    \caption{
        The $M_*-{\rm sSFR}$ distribution of BGS galaxies at $z < 0.2$. 
        sSFR is derived using $\overline{\rm SFR}$, average SFR over the last 1
        Gyr, inferred using PROVABGS. 
        The $M_*-{\rm sSFR}$ distribution is bimodal with star-forming galaxies
        lying on the star-forming sequence.
        We classify galaxies with ${\rm sSFR} > 10^{-11.2}\,{\rm yr}^{-1}$ as
        star-forming galaxies and ${\rm sSFR} < 10^{-11.2}\,{\rm yr}^{-1}$ as
        quiescent. 
    }\label{fig:sfq}
\end{center}
\end{figure}


\begin{figure}
\begin{center}
    \includegraphics[width=0.9\textwidth]{figs/psmf_bgs_bright_sfq.pdf} 
    \caption{
        The pSMF of star-forming (left) and quiescent (right) BGS Bright
        galaxies over $0.01 < z < 0.17$ in bins of $\Delta z = 0.04$. 
        Star-forming and quiescent galaxies are classified using an empirically
        determined ${\rm sSFR} = 10^{-11.2}{\rm yr}^{-1}$ cut. 
        We represent the uncertainties for the pSMF in the shaded regions and 
        the pSMFs above/below the $M_*$ completeness limits in solid/dashed
        lines.
        The pSMFs suggest a decline in massive, $M_* > 10^{11}M_\odot$ 
        star-forming galaxies and an increase in the quiescent galaxy population
        at $M_* < 10^{11}M_\odot$ at lower redshifts.
    }\label{fig:sfqsmf}
\end{center}
\end{figure}

\subsection{Star-Forming and Quiescent Galaxies in the BGS} \label{sec:sfq}
In addition to the pSMF of the full galaxy population, we can also examine the
pSMF of the star-forming and quiescent subpopulations using 
$\overline{\rm SFR}$, average SFR over the last 1 Gyr, inferred using PROVABGS. 
In Figure~\ref{fig:sfq}, we present the distribution of $M_*$ versus average
specific SFR, $\overline{\rm sSFR} = \overline{\rm SFR}/M_*$, for BGS Bright
(blue) and Faint (orange) galaxies at $z < 0.2$. 
The $M_*-{\rm sSFR}$ distribution of the BGS galaxies reveal a clear
bimodality with star-forming galaxies lying on the SFS and quiescent galaxies
lying $\gtrsim$1 dex below the sequence. 
Figure~\ref{fig:sfq} also confirms that BGS Faint galaxies have overall lower
$M_*$ than BGS Bright galaxies and are primarily star-forming galaxies. 
This is due to the fact that the $(z - W1)-1.2(g-r)+1.2$ color used to select
BGS Faint galaxies is a proxy for H$\alpha$ and H$\beta$ emission lines.  

To further examine the star-forming and quiescent galaxy populations, we
classify BGS Bright galaxies as star-forming or quiescent using a 
$\overline{\rm sSFR} = 10^{-11.2}\,{\rm yr}^{-1}$ cut. 
We determine this cut empirically based roughly on the sSFR of the ``green
valley'' between the SFS and the quiescent mode. 
We opt for a $\overline{\rm sSFR}$ cut rather than more sophisticated methods
in the literature~\citep[\emph{e.g.}][]{hahn2019, donnari2019} for simplicity. 
In Figure~\ref{fig:sfqsmf}, we present the pSMF of star-forming and quiescent
BGS Bright galaxies at $0.01 < z < 0.17$ in bins of $\Delta z = 0.04$.
The shaded regions represent the jackknife uncertainties for the pSMF. 
The solid lines represent the pSMFs above the completeness limit while the
dashed lines represent the pSMFs below the limit. 
The pSMF of quiescent galaxies suggest an increase in the number of galaxies
below $M_* < 10^{11}M_\odot$.
Meanwhile, the pSMF of star-forming galaxies shows a possible decline at the
massive end over $0.01 < z < 0.17$. 

\begin{figure}
\begin{center}
    \includegraphics[width=0.45\textwidth]{figs/qf_bgs_bright.pdf} 
    \caption{
        The quiescent fraction of BGS Bright galaxies over $0.01 < z < 0.17$ in
        bins of $\Delta z =0.04$.
        We present the uncertainties in the shaded region and only include  
        the quiescent fraction above the $M_*$ completeness limit. 
        The quiescent fractions increase with $M_*$ at all redshifts. 
        Furthermore, the quiescent fractions suggest an overall increase in the
        quiescent population with lower redshift.
    }\label{fig:qf}
\end{center}
\end{figure}

Next, we present the fraction of quiescent galaxies in BGS Bright as a function
of $M_*$ over $0.01 < z < 0.17$ in Figure~\ref{fig:qf}.
The quiescent fraction is derived by taking the ratio of the pSMFs of quiescent
galaxies over all galaxies and measured for each $\Delta z =0.04$ bin.
The shaded region represent the uncertainties derived from propagating the
jackknife uncertainties of the pSMFs. 
We focus on the quiescent fraction of BGS Bright galaxies above the $M_*$
completeness limit: $M_* > M_{\rm lim}$. 
At each redshift bin, the quiescent fraction increases with $M_*$ to $\sim$1 at
$M_*\sim10^{11.5}M_\odot$. 
The quiescent fraction also suggests an increase in the quiescent population
with redshift. 
Although the significant statistical uncertainties obfuscate a clear trend, 
the quiescent fraction evolution is in good qualitative agreement with previous
works~\citep[\emph{e.g.}][]{baldry2006, iovino2010, peng2010, hahn2015}.
Upcoming observations from the DESI main survey will increase the number of BGS
galaxies by >100$\times$ and enable precise comparisons of the quiescent
fraction measurements. 

% --- summary ---  
\section{Summary and Discussion} \label{sec:summary}



discussion of BGS in the DESI main survey


\todo{mention federico's paper as subsequent work with schetcher function fits} 

In subsequent work we will extend the hierarhical inference framework in 
this work to the SFR-$M_*$ distribution and present the probabilistic 
SFR-$M_*$ distribution and quiescent fraction. 



\section*{Acknowledgements}
It's a pleasure to thank . 
This work was supported by the AI Accelerator program of the Schmidt Futures
Foundation.

This research is supported by the Director, Office of Science, Office of High
Energy Physics of the U.S. Department of Energy under Contract No.
DE-AC02-05CH11231, and by the National Energy Research Scientific Computing
Center, a DOE Office of Science User Facility under the same contract;
additional support for DESI is provided by the U.S. National Science
Foundation, Division of Astronomical Sciences under Contract No. AST-0950945 to
the NSF's National Optical-Infrared Astronomy Research Laboratory; the Science
and Technologies Facilities Council of the United Kingdom; the Gordon and Betty
Moore Foundation; the Heising-Simons Foundation; the French Alternative
Energies and Atomic Energy Commission (CEA); the National Council of Science
and Technology of Mexico (CONACYT); the Ministry of Science and Innovation of
Spain (MICINN), and by the DESI Member Institutions:
\url{https://www.desi.lbl.gov/collaborating-institutions}.

The DESI Legacy Imaging Surveys consist of three individual and complementary
projects: the Dark Energy Camera Legacy Survey (DECaLS), the Beijing-Arizona
Sky Survey (BASS), and the Mayall z-band Legacy Survey (MzLS). DECaLS, BASS and
MzLS together include data obtained, respectively, at the Blanco telescope,
Cerro Tololo Inter-American Observatory, NSF’s NOIRLab; the Bok telescope,
Steward Observatory, University of Arizona; and the Mayall telescope, Kitt Peak
National Observatory, NOIRLab. NOIRLab is operated by the Association of
Universities for Research in Astronomy (AURA) under a cooperative agreement
with the National Science Foundation. Pipeline processing and analyses of the
data were supported by NOIRLab and the Lawrence Berkeley National Laboratory.
Legacy Surveys also uses data products from the Near-Earth Object Wide-field
Infrared Survey Explorer (NEOWISE), a project of the Jet Propulsion
Laboratory/California Institute of Technology, funded by the National
Aeronautics and Space Administration. Legacy Surveys was supported by: the
Director, Office of Science, Office of High Energy Physics of the U.S.
Department of Energy; the National Energy Research Scientific Computing Center,
a DOE Office of Science User Facility; the U.S. National Science Foundation,
Division of Astronomical Sciences; the National Astronomical Observatories of
China, the Chinese Academy of Sciences and the Chinese National Natural Science
Foundation. LBNL is managed by the Regents of the University of California
under contract to the U.S. Department of Energy. 
The complete acknowledgments can be found at
\url{https://www.legacysurvey.org/}.

The authors are honored to be permitted to conduct scientific research on
Iolkam Du’ag (Kitt Peak), a mountain with particular significance to the Tohono
O’odham Nation.

\appendix
\section{Spectroscopic Completeness} \label{sec:spec_comp}
Spectroscopic galaxy surveys, such as BGS, do not successfully measure the
redshift for all of the galaxies they target. 
As a result, this spectroscopic incompleteness must be accounted for when
measuring galaxy population statistics such as the SMF.  
In this appendix, we present how we estimate the spectroscopic incompleteness
for BGS and derive the weights we use to correct for its impact on the SMF. 

For BGS, spectroscopic incompleteness is primarily driven by fiber assignment
and redshift failures.  
DESI uses 10 fiber-fed spectrographs with 5000 fibers but targets more galaxies
than available fibers. 
For instance, the BGS Bright and Faint samples have $\sim 860$ and 
$530\,{\rm targets}/{\rm deg}^2$, respectively. 
For the 8 ${\rm deg}^2$ field-of-view of DESI, this roughly correspond to
11,000 targets, significantly more than the 5000 available fibers. 
DESI only measures the spectra of targets that are assigned fibers. 
In fact, of the 5000, a minimum of 400 ‘sky’ fibers are dedicated to measuring
the sky background for accurate sky subtraction and an additional 100 fibers
are assigned to standard stars for flux calibration~\cite{guy2022}.

Furthermore, each fiber is controlled by a robotic fiber positioner on the
focal plane. 
These positioners can rotate on two arms and be positioned within a circular
patrol region of radius 1.48 arcmin~\citep{schubnell2016, desi2016a,2022b, silber2022}.
Although the patrol regions of adjacent positioners slightly overlap, the
geometry of the positioners cause higher incompleteness in regions with high
target density~\citep{smith2019}.
To mitigate the incompleteness from the fiber assignment, BGS will observe its  
footprint with four passes.
With this strategy, BGS achieves $\sim$80\% fiber assignment
completeness~\citep{hahn_bgs}.

To estimate fiber assignment completeness, we run the fiber assignment
algorithm~\citep{raichoor2022} on BGS targets 128 separate times.
For each BGS galaxy, $i$, we count the total number of times out of 128 that
the galaxy is assigned a fiber: $N_{i, {\rm FA}}$. 
Then to correct for the fiber assignment incompleteness, we assign correction
weights
\begin{equation} \label{eq:w_fa}
    w_{i, \mathrm{FA}} = \frac{128}{N_{i, \mathrm{FA}}}
\end{equation}
to each BGS galaxy. 
{\color{red} explain what this means} 

\begin{figure}[h]
\begin{center}
    \includegraphics[width=0.5\textwidth]{figs/bgs_bright_rfib_tsnr2.pdf} 
    \caption{
        Redsift success rate of BGS Bright galaxies as a function of 
        $r_{\rm fiber}$ and TSNR2.
        TSNR2 is a statistic that quantifies the signal-to-noise ratio of the
        observed spectrum. 
        The color map represents the mean redshift success rate in each hexbin.
        We mark the TSNR2 bins (black dashed) that we use to separately fit the
        redshift success rate as a function of $r_{\rm fiber}$ using
        Eq.~\ref{eq:zsucc}.
        In each TSNR2 bin, redshift success decreases as $r_{\rm fiber}$
        increases. 
    }\label{fig:zfail0}
\end{center}
\end{figure}

Although we measure a spectrum for each galaxy assigned a fiber, we do not
accurately measure redshifts for every spectra. 
This redshift measurement failure significantly contributes to spectroscopic
incompleteness. 
For BGS, redshift failure of an observed galaxy spectrum depends mainly on
fiber magnitude and a statistic, TSNR2.
Fiber magnitude is the predicted flux of the BGS object within a 1.5\arcsec
diameter fiber; we use $r$-band fiber magnitude, $r_{\rm fiber}$.  
TSNR2 roughly corresponds to the signal-to-noise ratio of the spectrum and is 
the statistic used to calibrate the effective exposure times in DESI
observations (\todo{CITE}).

In Figure~\ref{fig:zfail0}, we present the redshift, $z$, success rate of BGS
Bright galaxies as a function of $r_{\rm fiber}$ and TSNR2.
In each hexbin, the color map represents the mean $z$-success rate. 
We include all hexbins with more than 2 galaxies. 
Overall, the $z$-success rate depends significantly on $r_{\rm fiber}$:
galaxies with fainter $r_{\rm fiber}$ have lower $z$-success rates. 
However, the $r_{\rm fiber}$ dependence itself varies in bins of TSNR2. 
We mark the edges of the bins in black dashed: $\log {\rm TSNR2} = 2.0, 2.5,
3.0, 3.5, 3.85$.
Within each of the TSNR2 bins, the $r_{\rm fiber}$ dependence of the
$z$-success rate does not vary significantly. 
In Figure~\ref{fig:zfail1}, we present the $z$-success rate of BGS Bright
galaxies as a function of $r_{\rm fiber}$ for each of the 6 TSNR2 bins.
We mark the range of TSNR2 in the bottom left of each panel. 
The errorbars represent the Poisson uncertainties of the $z$-success rate.

\begin{figure}
\begin{center}
    \includegraphics[width=0.5\textwidth]{figs/bgs_bright_rfib_tsnr2_zsuccess.pdf}
    \caption{
        Redshift success rates of BGS Bright galaxies  as a function of 
        $r_{\rm fiber}$ in 6 TSNR2 bins. 
        The error bars represent the poisson uncertainties.
        In each panel, we include the best-fit analytic (Eq.~\ref{eq:zsucc})
        approximation of the redshift success rate (dashed) derived from
        $\chi^2$ minimization. 
        We use this analytic approximation to calculate the galaxy weights to
        correct for spectroscopic incompleteness caused by failures to
        accurately measure redshifts from observed spectra.
    }\label{fig:zfail1}
\end{center}
\end{figure}

To correct for the effect of redshift failures, we include an additional
correction weight for each BGS galaxy: 
\begin{equation}
    w_{i, {\rm ZF}} = \frac{1}{f_{z-{\rm sucess}}(r_{{\rm fiber},i}, {\rm
    TSNR2}_i)}.
\end{equation} 
$f_{z-{\rm sucess}}(r_{{\rm fiber},i}, {\rm TSNR2}_i)$ is the $z$-success rate
as a function of $r_{\rm fiber}$ and TSNR2 of the galaxy. 
Galaxies with $f_(z-{\rm sucess} = 1.$ (100\% $z$-success) will have 
$w_{i, {\rm ZF}} = 1$ while galaxies with $f_(z-{\rm sucess} = 0.1$ 
(10\% $z$-success) will have $w_{i, {\rm ZF}} = 10$.
For $f_{z-{\rm sucess}}(r_{{\rm fiber},i}, {\rm TSNR2}_i)$, we fit the
following functional form for each TSNR2 bin: 
\begin{equation} \label{eq:zsucc}
    f_{z-{\rm success}}(r_{\rm fiber}) = \frac{1}{2} \bigg(1-{\rm erf}(c_0
    (r_{\rm fiber} - c_1))\bigg).
\end{equation}
In Figure~\ref{fig:zfail1}, we present the best-fit 
$f_{z-{\rm success}}(r_{\rm fiber})$ for each of the TSNR2 bins in dashed. 
The best-fit coefficients, $c_0, c_1$, are derived from $\chi^2$ minimization.
We repeat this procedure indepedently for BGS Bright galaxies as well as the
BGS Faint galaxies with $(z - W1) - 1.2(g - r) + 1.2 \ge 0$,
and BGS Faint galaxies with $(z - W1) - 1.2(g - r) + 1.2 < 0$.
We list the best-fit values in bins of TSNR2 for each of the samples in
Table~\ref{tab:zfail}. 

\begin{table} 
    \caption{Best-fit coefficients of the $z$-success rate as a function of
    $r_{\rm fiber}$ for different TSNR2 bins for BGS Bright and Faint samples.} 
    \begin{center}
        \begin{tabular}{lcc} \toprule
            TSNR2 range & $c_0$ & $c_1$ \\[3pt]
            \multicolumn{3}{c}{BGS Bright} \\
            \hline 
            $10^{1.5}   - 10^{2}$       & 0.443 & 19.7 \\ 
            $10^{2}     - 10^{2.5}$       & 0.668 & 21.3 \\ 
            $10^{2.5}   - 10^{3}$       & 0.888 & 22.1 \\ 
            $10^{3}     - 10^{3.5}$       & 0.822 & 22.4 \\ 
            $10^{3.5}   - 10^{3.85}$    & 0.698 & 23.3 \\ 
            $10^{3.85}  - 10^{5}$      & 0.465 & 21.8 \\ 
            \hline            
            \hline 
            \multicolumn{3}{c}{BGS Faint}\\
            \multicolumn{3}{c}{$(z - W1) - 1.2(g - r) + 1.2 \ge 0$} \\
            \hline 
            $10^{1.5}   - 10^{2.5}$     & 1.67  & 21.1 \\ 
            $10^{2.5}   - 10^{3}$       & 1.65  & 21.8 \\ 
            $10^{3}     - 10^{3.1}$     & 1.49  & 22.1 \\ 
            $10^{3.1}   - 10^{3.2}$     & 1.32  & 22.3 \\ 
            $10^{3.2}   - 10^{3.3}$     & 1.33  & 22.4 \\ 
            $10^{3.3}   - 10^{3.5}$     & 0.907 & 23.1 \\ 
            $10^{3.5}   - 10^{3.85}$    & 1.03  & 23.0 \\ 
            $10^{3.85}  - 10^{5}$       & 0.924 & 21.6 \\ 
            \hline 
            \hline 
            \multicolumn{3}{c}{BGS Faint}\\
            \multicolumn{3}{c}{$(z - W1) - 1.2(g - r) + 1.2 < 0$} \\
            \hline 
            $10^{2.5}   - 10^{3}$       & 1.48  & 20.9 \\ 
            $10^{3}     - 10^{3.1}$     & 2.40  & 21.2 \\ 
            $10^{3.1}   - 10^{3.2}$     & 1.30  & 21.8 \\ 
            $10^{3.2}   - 10^{3.3}$     & 1.27  & 22.0 \\ 
            $10^{3.3}   - 10^{3.5}$     & 1.83  & 21.6 \\ 
            $10^{3.5}   - 10^{3.85}$    & 0.798 & 22.9 \\ 
            $10^{3.85}  - 10^{5}$       & 1.29  & 20.6 \\ 
            \hline 
        \end{tabular} \label{tab:zfail}
    \end{center}
\end{table}
%logtsnr_min_em = [1.5, 2.5, 3., 3.1, 3.2, 3.3, 3.5, 3.85]
%logtsnr_max_em = [2.5, 3.0, 3.1, 3.2, 3.3, 3.5, 3.85, 5.0]
%c0s_faint_em = [1.67163017, 1.6478732, 1.49206734, 1.31762355, 1.3281991, 0.90687631, 1.02655811, 0.92365139]
%c1s_faint_em = [21.14342871, 21.79206685, 22.09573976, 22.33046202, 22.40893427, 23.09747128, 23.02003987, 21.6161412]

%logtsnr_min_noem = [2.5, 3., 3.1, 3.2, 3.3, 3.5, 3.85]
%logtsnr_max_noem = [3.0, 3.1, 3.2, 3.3, 3.5, 3.85, 5.0]
%c0s_faint_noem = [1.47794757, 2.40488178, 1.30053995, 1.26645044, 1.83089418, 0.79824946, 1.2882063]
%c1s_faint_noem = [20.92275801, 21.16670108, 21.80779297, 21.98731592, 21.66870409, 22.89227465, 20.63296212] 


% --- references ---
% https://desi.lbl.gov/trac/wiki/ClusteringWG/LSScat/DA02main/current_version#clusteringfiles


\begin{figure}
\begin{center}
    \includegraphics[width=0.5\textwidth]{figs/jackknife_fields.pdf} 
    \caption{
        The RA and Dec of the 12 jackknife fields of the BGS One-Percent Survey
        used to estimate the uncertainties on the SMF from sample variance. 
        We mark each field with a distinct color. 
    }\label{fig:jack}
\end{center}
\end{figure}


\section{Uncertainties on the SMF} \label{sec:uncert}
We estimate the uncertainties of the SMF from sample variance using the
standard jackknife technique. 
This involves splitting our BGS sample into subsamples and then estimating
uncertainties using the subsample-to-subsample variations:  
\begin{equation} \label{eq:jack} 
    \sigma_\Phi = \left(\frac{N_{\rm jack}-1}{N_{\rm jack}}
    \sum\limits_{k=1}^{N_{\rm jack}} (\Phi_k - \Phi)^2 \right).
\end{equation} 
$N_{\rm jack}$ is the number of jackknife subsamples and $\Phi_k$ represents
the SMF estimated from the BGS galaxies excluding the jackknife subample $k$. 
In this work, we split the BGS sample into 12 jackknife fields based on the
angular positions of galaxies. 
We present the jackknife fields in Figure~\ref{fig:jack} with distinct colors. 

\section{Stellar Mass Completeness} \label{sec:mscomp}
In this appendix, we describe how we derive $M_{\rm lim}$, the $M_*$ limit
above which our BGS Bright sample is complete. 
Although there are various methods for estimating $M_{\rm lim}$ in the
literature, \emph{e.g.} based on estimating the mass-to-light
ratio~\citep{pozzetti2010, moustakas2013}, we adopt a simple approach that
takes advantage of the fact that BGS Bright is a magnitude-limited sample. 

\begin{figure}
\begin{center}
    \includegraphics[width=0.45\textwidth]{figs/psmf_logMstar_comp_demo.pdf}
    \caption{
        The $M_*$ distribution of BGS Bright galaxies with $0.01 < z < 0.03$
        (blue) and the $M_*$ distribution of same set of galaxies that would
        remain in the BGS Bright magnitude limit if they were redshifted to 
        $z' = z + 0.02$: $r' < 19.5$. 
        We set the stellar mass completeness limit, $M_{\rm lim}$, for $0.01 <
        z < 0.05$ to the $M_*$ where more than 10\% of galaxies are excluded in
        the latter distribution. 
    }\label{fig:ms_comp0}
\end{center}
\end{figure}

To derive $M_{\rm lim}$ in redshift bins of width $\Delta z=0.04$, we first
split the galaxy sample into narrower bins of $\Delta z/2$. 
For each narrower redshift bin, $i \Delta z/2 < z < (i+1) \Delta z/2$, we take
all the best-fit {\sc PROVABGS} SEDs from all  galaxies in the bin and
artificially redshift it to $z' = z + \Delta z/2$:
\begin{equation}
    f_\lambda' = f_\lambda \frac{d_L(z)^2}{d_L(z')^2}.
\end{equation} 
$d_L(z)$ reprents the luminosity distance at redshift $z$. 
Afterward, we calculate the $r$-band magnitudes, $r'$, for $f_\lambda'$ and
impose the $r' < 19.5$ magnitude limit of the BGS Bright. 
We then compare the $M_*$ distribution of all the galaxies in 
$i \Delta z/2 < z < (i+1) \Delta z/2$ to the galaxies in 
$i \Delta z/2 < z < (i+1) \Delta z/2$ with $r' < 19.5$.
For instance,  we present the $M_*$ distributions of all BGS
Bright galaxies in $0.01 < z < 0.03$ (blue) and the BGS Bright galaxies in
$0.01 < z < 0.03$ with $r' < 19.5$ (orange) in Figure~\ref{fig:ms_comp0}.

\begin{table} 
    \caption{Stellar mass completeness limit, $M_{\rm lim}$ for redshift bins
    of width $\Delta z = 0.04$.} 
    \begin{center}
        \begin{tabular}{lc} \toprule
            $z$ range & $\log_{10} M_{\rm lim}$ \\[3pt]
            \hline 
            $0.01 - 0.05$   & 8.975 \\ 
            $0.05 - 0.09$   & 9.500 \\ 
            $0.09 - 0.13$   & 10.20 \\ 
            $0.13 - 0.17$   & 10.38 \\ 
            $0.17 - 0.21$   & 10.72 \\ 
            \hline            
\end{tabular} \label{tab:mscomp}
\end{center}
\end{table}

\begin{figure}[h]
\begin{center}
    \includegraphics[width=0.6\textwidth]{figs/psmf_logMstar_comp_z.pdf}
    \caption{
        $M_*$ and redshift relation of BGS Bright galaxies in the EDR (black)
        and the galaxies within the stellar mass completeness limit ($M_* <
        M_{\rm lim}$; blue). 
        $M_{\rm lim}$ is derived in redshift bins of width $\Delta z = 0.04$. 
        The lowest redshift bin ($0.01 < z < 0.05$) is complete down to 
        $M_* < 10^9 M_\odot$. 
    }\label{fig:ms_comp1}
\end{center}
\end{figure}

Since galaxies become fainter when they are placed at higher redshifts,
\emph{i.e.} $r' > r$, the $r' < 19.5$ sample has fewer low $M_*$ galaxies. 
We determine the $M_*$ at which, more than 10\% of galaxies are excluded in the
$r' < 19.5$ sample (black dashed) and set this limit as $M_{\rm lim}$ for the
redshift bins: $0.01 < z < 0.05$.
Our procedure for deriving $M_{\rm lim}$ takes advantage of the fact that
galaxy samples at lower redshifts are complete down to lower $M_*$ than at
higher redshifts. 
We repeat this procedure for all the $\Delta z = 0.04$ redshift bins that we
use to measure the SMF.
In Table~\ref{tab:mscomp}, we list $M_{\rm lim}$ values for each of the
redshift bins. 
Furthermore, we present the $M_*$ and redshift relation of BGS Bright galaxies
(black) and the stellar mass complete sample (blue) in
Figure~\ref{fig:ms_comp1}. 


\bibliographystyle{mnras}
\bibliography{psmf} 
\end{document}
