\section{Introduction} \label{sec:intro} 
% brief review of scaling relations and the importance of population statistics
% list of all the population statistics paper from SDSS and GAMA. 
With large galaxy surveys, such as the Sloan Digital Sky
Survey~\citep[SDSS;][]{york2000}, Galaxy and Mass Assembly
survey~\citep[GAMA;][]{driver2011}, and 
PRIsm MUlti-object Survey~\citep[PRIMUS;][]{coil2011}, 
% maybe break this up into two pieces  
the galaxy population can now largely be characterize with a small number of
scaling relations and population 
statistics~\citep[see][for a review]{blanton2009}.
The stellar mass function (SMF), for instance, precisely characterizes the
overall stellar mass, $M_*$, distribution of galaxies and its
evolution across cosmic history~\citep{li2009, marchesini2009, moustakas2013,
muzzin2013, leja2019a, driver2022}.

%Marchesini et al. 2009; Muzzin et al. 2013; Ilbert et al. 2013; Moustakas et al. 2013; Tomczak et al. 2014; Grazian et al. 2015; Song et al. 2016; Davidzon et al. 2017; Wright et al. 2018).

The relationship between the stellar masses and star formation rates of
galaxies reveal a bimodality in the galaxy population with star-forming
galaxies lying on a tightly correlated ``star formation
sequence''~\citep[SFS;][]{noeske2007, daddi2007, salim2007, speagle2014,
hahn2019}.
From $z\sim 2$, there is an overall decline in star formation rates of galaxies
in the SFS. 
This is accompanied by an increase in the fraction of quiescent galaxies caused
by the quenching of star formation in some galaxies~\citep{kauffmann2003a,
blanton2003, baldry2006, taylor2009}. 
Additional scaling relations among galaxy propreties such as the
mass-metallicity relation~\citep{tremonti2004} or \todo{add more scaling
relations here} have also been firmly established. 
They connect \todo{some physical insight} 
More precise and accurate measurements of the statistical distributions of the
properties for galaxy populations at different cosmic epochs have the potential
to shed further light on galaxy formation and evolution. 

For one, they have the potential to reveal new trends among galaxies undetected
by previous observations and open new discovery space.
They can also be used to test galaxy formation models spanning 
empirical models~\citep[\emph{e.g.} {\sc UniverseMachine};][]{behroozi2019}, 
semi-analytic models~\citep[\emph{e.g.}][]{benson2012, henriques2015,
somerville2015a}, and 
hydrodyanmical simulations~\citep[see][for a review]{somerville2015a}. 
Empirical models, for example, have been used to measure the timescale of 
timescale of star formation quenching~\citep{wetzel2013, hahn2017, tinker2017}
or the dust content of galaxies~\citep{hahn2021}. 

Furthermore, observations have already been used to infer parameters that
dictate the physical processes in semi-analytic
models~\citep[\emph{e.g.}][]{henriques2009, lu2014, henriques2015} 
Although full parameter exploration is currently computationally prohibitively
for hydrodynamical simulations, they have been extensively compared to
observations: \emph{e.g.}~\cite{genel2014, dave2017a, trayford2017, dickey2021,
donnari2021}.
Soon machine learning techniques for accelerating and emulating simulations
will enable us to go beyond such comparisons and broadly explore parameter
space and galaxy formation
models~\citep[\emph{e.g.}][]{villaescusa-navarro2022a, jamieson2022}.
While many different approaches are available for expanding our understanding
of galaxies, they all require statistically powerful galaxy samples with
well controlled systematics and well understood selection functions. 

One survey that will provide galaxy samples with unprecedented statistical power
is the Dark Energy Spectrscopic
Instrument~\citep[DESI;][]{desicollaboration2016, desicollaboration2016a,
abareshi2022}. 
Over its 5 year operation, DESI will observe galaxy spectra using the 4-meter
Mayll telescope at Kitt Peak National Observatory with a focal plane filled
with 5000 robotically-actuated fibers that direct the light to ten optical
spectrographs.
It will observe $\sim$40 million galaxy spectra over $360 < \lambda < 980$ nm
with spectral resolution of $2000 < \lambda/\Delta \lambda < 5500$ over 
${\sim}14,000~{\rm deg}^2$, a third of the sky.
In addition, DESI galaxies will also have photometry from the Legacy Imaging
Surveys Data Release 9~\citep[LS;][]{dey2019}. 
LS is a combination of three public projects (Dark Energy Camera Legacy Survey,
Beijing-Arizona Sky Survey, and Mayall $z$-band Legacy Survey) that jointly
imaged the DESI footprint in three optical bands ($g$, $r$, and $z$). 
DESI began observing its main survey in May 14, 2021. 

As part of the survey, DESI is conducting the Bright Galaxy
Survey~\citep[BGS;][]{hahn2022c}.
BGS spans the same 14,000${\rm deg}^2$ footprint and will include low redshift
$z< 0.6$ galaxies that can be observed during bright time, when the night sky
is ${\sim}2.5\times$ brighter than nominal dark conditions,
BGS will provide two galaxy samples: the BGS Bright sample, a $r < 19.5$
magnitude-limited sample of ${\sim}10$ million galaxies, and the BGS Faint
sample, a fainter $19.5 < r < 20.175$ sample of ${\sim 5}$ million galaxies
selected using a surface brightness and color. 
The selection and completeness of the BGS samples are characterized in detail
in \cite{hahn2022c}. 
Compared to the seminal SDSS main galaxy survey, BGS will provide a galaxy
sample two magnitudes deeper, over twice the sky, and double the median
redshift $z{\sim}0.2$. 
It will observe a broader range of galaxies than previous surveys  and provide
an opportunity to measure galaxy population statistics with unprecendented
precision.

BGS will also be accompanied by a value-added catalog: the Probabilistic
Value-Added BGS~\citep[PROVABGS;][]{hahn2022, kwon2022}.  
For every BGS galaxy, PROVABGS will provide physical properties such as stellar
mass ($M_*$), average star formation rate ($\overline{\rm SFR}$), stellar
metallicity ($Z_*$), stellar age ($t_{\rm age}$), and dust content. 
These galaxy properties will be from state-of-the-art Spectral Energy
Distribution (SED) modeling of both DESI photometry and spectroscopy in a full
Bayesian inference framework. 
The SED model is designed to minimize model misspecification by using a highly
flexible non-parameteric star formation and metallicity histories as well as a
flexible dust attenuation model.
Furthermore, the properties will be inferred using a fully Bayesian inference
framework and provide statistically rigorous estimates of uncertainties and
degeneracies among the properties.  
Ultimately, PROVABGS will provide consistently measured galaxy properties that
will enable analyses to fully take advantage of the statistical power of BGS
with new techniques and approaches. 

A key application for PROVABGS will be measuring population statistics using 
statistically rigorous methodology that correctly propagates the uncertainties
in galaxy property measurements. 
Current population statistics are by and large derived from simply binning
best-fit point estimates of galaxy properties. 
\cite{malz2020} demonstrated, in the context of inferring redshif distributions
from individual photometric redshift measurements, that using point estimates
is statistically incorrect and can lead to biased redshift distributions. 
Simiarly, the standard approach can also lead to biased population statistics. 

Instead, we can estimate population statistics from combining individual
PROVABGS posteriors of galaxy properties using population inference in a
hierarchical Bayesian framework~\citep[\emph{e.g.}][]{hogg2010,
foreman-mackey2014, baronchelli2020}.
This approach correctly propagates the uncertainties in the galaxy properties 
from the individual posteriors of galaxies. 
As a result, they significantly improve the accuracy of population statistics
measurements and will enable more accurate measurements of key galaxy scaling
relations. 
In this work, we present the first of such population statistic measurement for
BGS: the probabilistic stellar mass function (pSMF). 

In particular, we present the pSMF of BGS galaxies observed during the DESI
One-Percent Survey, a survey validation program conducted before the main
survey operations. 
We also present the statistical methodology for the population inference as
well as the our methods for accounting for observational incompleteness. 
We begin in Section~\ref{sec:edr} with an overview of the BGS galaxies observed
during the DESI One-Percent Survey. 
Then, in Section~\ref{sec:provabgs}, we briefly summarize the PROVABGS SED
modeling framework used to infer the physical properties of the BGS galaxies.
Afterwards, we present the pSMF inferred from the BGS observations in
Section~\ref{sec:results}. 
We summarize and discuss our results in Section~\ref{sec:summary}.
Throughout this work, we assume AB magnitudes and a flat $\Lambda$CDM cosmology
described by the final Planck results~\citep{planckcollaboration2014a}: .
$\Omega_m = 0.307$, $\Omega_b = 0.0483$, $H_0=67.8\,{\rm km\,s^{-1} Mpc^{-1}}$,
$A_s=2.19\times10^{-9}$, $n_s=0.9635$
