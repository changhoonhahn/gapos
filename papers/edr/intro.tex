\section{Introduction} \label{sec:intro} 
% brief review of scaling relations and the importance of population statistics
% list of all the population statistics paper from SDSS and GAMA. 
With large galaxy surveys, such as the Sloan Digital Sky
Survey~\citep[SDSS;][]{york2000}, Galaxy and Mass Assembly
survey~\citep[GAMA;][]{driver2011}, and 
PRIsm MUlti-object Survey~\citep[PRIMUS;][]{coil2011}, 
the galaxy population can now largely be characterize with a small number of
scaling relations and population 
statistics~\citep[see][for a review]{blanton2009}.
For instance, the stellar mass function (SMF) precisely characterizes the
overall stellar mass, $M_*$, distribution of galaxies and its
evolution~\citep{li2009, marchesini2009, moustakas2013, muzzin2013, leja2019a,
driver2022}.

%Marchesini et al. 2009; Muzzin et al. 2013; Ilbert et al. 2013; Moustakas et al. 2013; Tomczak et al. 2014; Grazian et al. 2015; Song et al. 2016; Davidzon et al. 2017; Wright et al. 2018).


Also, the relationship between the stellar masses and star formation rates of
galaxies reveal a bimodality in the galaxy population with star-forming
galaxies forming the ``star formation sequence''~\citep{noeske2007, daddi2007,
salim2007}.


or quiescent
fraction~\citep{kauffmann2003a, blanton2003, baldry2006, taylor2009}, and their
evolution are now well understood. 
Many global scaling relations of galaxy propreties such as the mass-metallicity
relation~\citep{tremonti2004} or

also been firmly established. 

More precise and accurate measurements of the statistical distributions of the
properties for galaxy populations at different cosmic epochs have the potential
to shed further light on galaxy formation and evolution. 

For one, they have the potential to reveal new trends among galaxies undetected
by previous observations and open new discovery space.
They can also be used to test galaxy formation models spanning 
empirical models~\citep[\emph{e.g.} {\sc UniverseMachine};][]{behroozi2019}, 
semi-analytic models~\citep[\emph{e.g.}][]{benson2012, henriques2015,
somerville2015}, and 
hydrodyanmical simulations~\citep[see][for a review]{somerville2015a}. 
Empirical models, for example, have been used to measure the timescale of 
timescale of star formation quenching~\citep{wetzel2013, hahn2017, tinker2017}
or the dust content of galaxies~\citep{hahn2021}. 

Furthermore, observations have already been used to infer parameters that
dictate the physical processes in semi-analytic
models~\citep[\emph{e.g.}][]{henriques2009, lu2014, henriques2015} 
Although full parameter exploration is currently computationally prohibitively
for hydrodynamical simulations, they have been extensively compared to
observations: \emph{e.g.}~\cite{genel2014, dave2017a, trayford2017, dickey2021,
donnari2021}.
Soon machine learning techniques for accelerating and emulating simulations
will enable us to go beyond such comparisons and broadly explore parameter
space and galaxy formation
models~\citep[\emph{e.g.}][]{villaescusa-navarro2021}.
While many different approaches are available for expanding our understanding
of galaxies, they all require statistically powerful galaxy samples with
well controlled systematics and well understood selection functions. 

One survey that will provide galaxy samples with unprecedented statistical power
is the Dark Energy Spectrscopic
Instrument~\citep[DESI;][]{desicollaboration2016, desicollaboration2016a}. 
Over its 5 year operation, DESI will observe galaxy spectra using the 4-meter
Mayll telescope at Kitt Peak National Observatory with a focal plane filled
with 5000 robotically-actuated fibers that direct the light to ten optical
spectrographs.
It will observe $\sim$40 million galaxy spectra over $360 < \lambda < 980$ nm
with spectral resolution of $2000 < \lambda/\Delta \lambda < 5500$ over 
${\sim}14,000~{\rm deg}^2$, a third of the sky.
In addition, DESI galaxies will also have photometry from the Legacy Imaging
Surveys Data Release 9~\citep[LS;][]{dey2019}. 
LS is a combination of three public projects (Dark Energy Camera Legacy Survey,
Beijing-Arizona Sky Survey, and Mayall $z$-band Legacy Survey) that jointly
imaged the DESI footprint in three optical bands ($g$, $r$, and $z$). 
DESI began its main observing in May 14, 2021. 

As part of the survey, DESI is conducting the Bright Galaxy
Survey~\citep[BGS;][]{hahn2022}.
BGS spans the same 14,000${\rm deg}^2$ footprint and will include low redshift
$z< 0.6$ galaxies that can be observed during bright time, when the night sky
is ${\sim}2.5\times$ brighter than nominal dark conditions,
BGS will provide two galaxy samples: the BGS Bright sample, a $r < 19.5$
magnitude-limited sample of ${\sim}10$ million galaxies, and the BGS Faint
sample, a fainter $19.5 < r < 20.175$ sample of ${\sim 5}$ million galaxies
selected using a surface brightness and color. 
The selection and completeness of the BGS samples are characterized in detail
in \cite{hahn2023}. 
Compared to the seminal SDSS main galaxy survey, BGS will provide a galaxy
sample two magnitudes deeper, over twice the sky, and double the median
redshift $z{\sim}0.2$. 
It will observe a broader range of galaxies than previous surveys  and provide
an opportunity to measure galaxy population statistics with unprecendented
precision.

BGS will also be accompanied by a value-added catalog: the Probabilistic
Value-Added BGS~\citep[PROVABGS;][]{hahn2022, kwon2022}.  
For every BGS galaxy, PROVABGS will provide physical properties such as stellar
mass ($M_*$), average star formation rate ($\overline{\rm SFR}$), stellar
metallicity $Z$, stellar age $t_{\rm age}$, and dust content. 
These galaxy properties will be inferred using state-of-the-art Spectral Energy
Distribution (SED) modeling of both DESI photometry and spectroscopy in a full
Bayesian inference framework. 
Ultimately, PROVABGS will provide consistently measured galaxy properties that
will enable analyses to fully take advantage of the statistical power of BGS
with new techniques and approaches. 

A key application for PROVABGS will be measuring population statistics in a 
statistically correct  
% a statistically rigorous probabilistic graphical model of redshift-dependent photometry, which correctly propagates the redshift uncertainty information

emphasize the Bayesian posteriors here 

In this work, we present the first of such population statistic measurement for
BGS: the probabilistic stellar mass function (pSMF). 

{\color{red} why this work?} 
In this work, we present the pSMF for galaxies in the Bright Galaxy Survey
observed as part of the DESI One-Percent Survey, a survey validation program
conducted before the main survey operations. 

Furthermore, in this work we present the statistical methodology as well as the
methodology for accounting for observational incompleteness. 

We begin in Section~\ref{sec:edr} with an overview of the BGS galaxies observed
during the DESI One-Percent Survey. 
Then, in Section~\ref{sec:provabgs}, we briefly summarize the PROVABGS SED
modeling framework used to infer the physical properties of the BGS galaxies.
Afterwards, we present the pSMF inferred from the BGS observations in
Section~\ref{sec:results}. 
We summarize and discuss our results in Section~\cite{sec:summary}.
Throughout the 
We use XXXXXX cosmology.
