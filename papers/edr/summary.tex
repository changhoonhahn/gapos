\section{Summary and Discussion} \label{sec:summary}
Over its five year operation, starting on May 2021, the DESI Bright Galaxy
Survey (BGS) will observe the spectra of $\sim$15 million galaxies out to 
$z < 0.6$ over 14,000 ${\rm deg}^2$.  
BGS will produce two galaxy samples: a $r < 19.5$ magnitude-limited BGS Bright
sample and a fainter $19.5 < r < 20.175$ surface brightness and color selected
BGS Faint sample. 
Compared to the SDSS main galaxy survey, BGS galaxy samples will be two
magnitudes deeper, over twice the sky, and double the median redshift
$z\sim0.2$. 
Furthermore, each galaxy in BGS will have measurements of its physical
properties (\emph{e.g.} $M_*$, SFR, $Z_*$, $t_{\rm age}$) derived using
state-of-the-art SED mdoeling of DESI spectrophotometry from PROVABGS. 

The BGS samples will include diverse galaxy subpopulations that can be used to
reveal new trends among galaxies that were preivously undetectable and open
new discovery space. 
They will also provide statistical power to measure scaling relations and
population statistics to characterize the global galaxy population and test
galaxy formation models with unprecedented precision.

In this work, we showcase the potential of BGS by presenting the probabilistic
stellar mass function (pSMF) using only $\sim$250,000 BGS galaxies observed
during one of DESI's survey validation program.
We present the hierarhical population inference framework that enable us to
accurately propgate uncertainties on galaxy properties ($M_*$) to population
statistics (pSMF). 
We also describe how we account for selection effect and incompleteness
(Appendix~\ref{sec:spec_comp}).  


All of the pSMFs presented in this work are measured from BGS galaxies observed during the DESI One-Percent Survey. 
DESI has already completed nearly two years of observations.
\todo{current status of BGS} 


\todo{subsequent works}
federico's paper as subsequent work with schetcher function fits

detailed stellar mass comparison project. 

Furthermore, in subsequent work we will extend the hierarhical population
inference framework to the SFR-$M_*$ distribution and present the
probabilistic SFR-$M_*$ distribution and quiescent fraction. 

