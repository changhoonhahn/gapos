\section{Summary and Discussion} \label{sec:summary}
Over its five year operation, starting on May 2021, the DESI Bright Galaxy
Survey (BGS) will observe the spectra of $\sim$15 million galaxies out to 
$z < 0.6$ over 14,000 ${\rm deg}^2$.  
BGS will produce two main galaxy samples: a $r < 19.5$ magnitude-limited BGS
Bright sample and a fainter $19.5 < r < 20.175$ surface brightness and color
selected BGS Faint sample. 
Compared to the SDSS main galaxy survey, the BGS galaxy samples will be over
two magnitudes deeper, twice the sky, and double the median redshift
$z\sim0.2$. 
They will include diverse galaxy subpopulations that have the potential to
reveal new trends among galaxies that were preivously undetectable and open new
discovery space. 

In addition, each galaxy in BGS will have measurements of its detailed physical
properties (\emph{e.g.} $M_*$, SFR, $Z_*$, $t_{\rm age}$) from PROVABGS. 
These properties will be inferred from DESI spectrophotometry using
state-of-the-art  SED modeling in a fully Bayesian inference framework. 
PROVABGS will provide statistically rigorous estimates of uncertainties and
degeneracies among the properties.
With these measurements, BGS will be a consistent and statistical powerful
galaxy sample to measure scaling relations and population statistics to
characterize the global galaxy population and test galaxy formation models
with unprecedented precision.

In this work, we showcase the potential of BGS by presenting the probabilistic
stellar mass function (pSMF) using $\sim$250,000 BGS galaxies observed solely
during one of DESI's survey validation program.
The pSMF are derived using a hierarhical population inference framework that 
statistically rigorously propgates uncertainties on $M_*$ and provide improve
estimates of the SMF at the lowest and highest $M_*$ regimes. 
We also describe how we account for selection effect and incompleteness in the
BGS observations (Appendix~\ref{sec:spec_comp}).  
Overall, we find good agreement between our pSMF and previous SMF measurements
in the literature. 
We also examine the pSMF of the star-forming and quiescent galaxy population
classified using a simple $\overline{\rm sSFR} = 10^{-11.2}{\rm yr}^{-1}$ cut
and find qualitative agreement with previous works. 

This work is first of a series of papers that will present population
statistics for BGS galaxies using PROVABGS. 
For the pSMF in this work, we used a flexible GMM to provide a non-parametric
measurement of the SMF.
In a subsequent work, Speranza~\etal~(in prep.), we will present the pSMF of BGS
measured using a parametric model with continuous redshift evolution. 
In another work, we will present in depth comparison $M_*$ measured using
different methodologies and assumptions. 
Lastly, the hierarchical population inference framework presented in this work
can be extended to population statistics beyond the SMF. 
We will extend the framework to the SFR-$M_*$ distribution and present the
probabilistic SFR-$M_*$ distribution and quiescent fraction in future work. 

All of the pSMFs presented in this work are measured from BGS galaxies observed
from April to May of 2021 during the DESI One-Percent Survey.
Since then, DESI has already completed nearly two years of observations.
As of writing (March 2023), DESI has observed nearly 20 million galaxy spectra
in total and nearly 9 million BGS galaxy spectra.  
With three out of the five years of operation remaining, BGS has completed
completed $\sim$60\% of its observations and is ahead of schedule.  
DESI observations will be publicly released periodically, starting with the
Early Data Release (EDR) later this year. 
The EDR will include observations from the One-Percent Survey used in this
this work. 
An accompanying PROVABGS catalog will be released with each data release.
