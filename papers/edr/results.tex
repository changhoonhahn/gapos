\section{Results} \label{sec:results}
We are interested in estimating the SMF of BGS galaxies from their individual
marginalized posteriors, $p(M_* \given {\bfi X_i})$, derived using
PROVABGS~(Section~\ref{sec:provabgs}). 

we're going to do population inference in a hierarchical bayesian framework and
use a normalizing flow.
%We can infer the population hyperparameters, μ∆θ and σ∆θ , using a hierarchical Bayesian framework (e.g. Hogg et al. 2010; Foreman-Mackey et al. 2014; Baronchelli et al. 2020).

why? because it produces unbiased inference. 

why do we use normalizing flows? 

We follow the same approach as \cite{hahn2022} to estimate:
\begin{align}\label{eq:popinf}
p(\phi \given \{{\bfi X_i}\}) 
    =&~\frac{p(\phi)~p( \{{\bfi X_i}\} \given \phi)}{p(\{{\bfi X_i}\})}\\
    =&~\frac{p(\phi)}{p(\{{\bfi X_i}\})}\int p(\{{\bfi X_i}\} \given \{\theta_i\})~p(\{\theta_i\} \given \phi)~{\rm d}\{\theta_i\}.\\
    =&~\frac{p(\phi)}{p(\{{\bfi X_i}\})}\prod\limits_{i=1}^N\int p({\bfi X_i} \given \theta_i)~p(\theta_i \given \phi)~{\rm d}\theta_i\\
    =&~\frac{p(\phi)}{p(\{{\bfi X_i}\})}\prod\limits_{i=1}^N\int \frac{p(\theta_i \given {\bfi X_i})~p({\bfi X_i})}{p(\theta_i)}~p(\theta_i \given \phi)~{\rm d}\theta_i\\
    =&~p(\phi)\prod\limits_{i=1}^N\int \frac{p(\theta_i \given {\bfi X_i})~p(\theta_i \given \phi)}{p(\theta_i)}~{\rm d}\theta_i. 
\intertext{
    We estimate the integral using $S_i$ Monte Carlo samples from the
    individual posteriors $p(\theta_i \given {\bfi X_i})$: 
}
    \approx&~p(\phi)\prod\limits_{i=1}^N\frac{1}{S_i}\sum\limits_{j=1}^{S_i}
    \frac{p(\theta_{i,j} \given \phi)}{p(\theta_{i,j})}.
\end{align} 

BGS provides two samples: BGS Bright and Faint. 
Galaxies in BGS Bright are selected based on a $r < 19.5$ flux limit, while
the ones in BGS Faint are selected based on a fiber-magnitude and color limit
and $r < 20.0175$ flux limit. 
Since neither of these samples are volume-limited and complete as a function
of $M_*$, we must include the selection effect when estimating the SMF. 
We do this by including weights derived from $z^{\rm max}$, the maximum
redshift that galaxy $i$ could be placed and still be included in the BGS
samples. 
We derive $z^{\rm max}_i$ for every galaxy using by redshifting the SED
predicted by the best-fit parameters. 
We then derive $V^{\rm max}_i$, the comoving volume out to $z^{\rm max}_i$, and
weights $w_i = w_{i, {\rm comp}}/V^{\rm max}_i$. 
We modify Eq.~\ref{eq:popinf} to include $w_i$: 
\begin{align}
p(\phi \given \{{\bfi X_i}\}) 
    \approx&~\frac{p(\phi)}{\prod\limits_{i=1}^N p({\bfi X_i})^{w_i}} 
    \prod\limits_{i=1}^N \left(\int p({\bfi X_i} \given \theta_i)~p(\theta_i \given \phi)~{\rm d}\theta_i \right)^{w_i} \\ 
    \approx&~\frac{p(\phi)}{\prod\limits_{i=1}^N p({\bfi X_i})^{w_i}} 
    \prod\limits_{i=1}^N \left( \sum\limits_{j=1}^{S_i}
    \frac{p(\theta_{i,j} \given \phi)}{p(\theta_{i,j})} \right)^{w_i} \\
    \approx&~\frac{p(\phi)}{\prod\limits_{i=1}^N p({\bfi X_i})^{w_i}} 
    \prod\limits_{i=1}^N \left( \sum\limits_{j=1}^{S_i}
    \frac{q_\phi(\theta_{i,j})}{p(\theta_{i,j})} \right)^{w_i}.
\end{align} 

In practice, we do not derive the full posterior 
$p(\phi \given \{{\bfi X_i}\})$. 
Instead we derive the maximum a posteriori (MAP) hyperparameter 
$\phi_{\rm MAP}$ that maximizes $p(\phi \given \{{\bfi X_i}\})$ or 
$\log p(\phi \given \{{\bfi X_i}\})$.
We expand, 
\begin{align}
\log p(\phi \given \{{\bfi X_i}\}) 
    \approx&~\log p(\phi) + % \sum\limits_{i=1}^N w_i \log w_i + 
    \sum\limits_{i=1}^N w_i \log \left(\sum\limits_{j=1}^{S_i} \frac{q_\phi(\theta_{i,j})}{p(\theta_{i,j})} \right).
\end{align} 
Since the first two terms are constant, we derive $\phi_{\rm MAP}$ by
maximizing 
\begin{equation}
    \max_\phi~~\sum\limits_{i=1}^N w_i \log \left(\sum\limits_{j=1}^{S_i} \frac{q_\phi(\theta_{i,j})}{p(\theta_{i,j})} \right).
\end{equation}
We use {\sc Adam} optimizer and determine the architecture of the normalizing
flow through experimentation.  

paragraph summarizing the weights that we use: vmax, spectroscopic completeness
weights, 


%\approx&~\log p(\phi) - 
%\log \prod\limits_{i=1}^N p({\bfi X_i})^{w_i} + 
%\log \prod\limits_{i=1}^N \left(\int p({\bfi X_i} \given \theta_i)~p(\theta_i \given \phi)~{\rm d}\theta_i \right)^{w_i} \\
%\approx&~\log p(\phi) - 
%\sum\limits_{i=1}^N w_i \log p({\bfi X_i}) + 
%\sum\limits_{i=1}^N w_i \log \left(\int p({\bfi X_i} \given \theta_i)~p(\theta_i \given \phi)~{\rm d}\theta_i \right) \\
%\approx&~\log p(\phi) + 
%\sum\limits_{i=1}^N w_i \log \left(\frac{1}{w_i} \sum\limits_{j=1}^{S_i} w_{i,j} \frac{p(\theta_{i,j} \given \phi)}{p(\theta_{i,j}} \right) \\


%\subsection{Targeting Completeness} \label{sec:ts}
% https://desi.lbl.gov/trac/wiki/ClusteringWG/LSScat/SV3/version2.1/fulldat
% https://desi.lbl.gov/trac/wiki/ClusteringWG/LSScat/SV3/version2.1/fullran


\subsection{The Probabilistic Stellar Mass Function} \label{sec:psmf}

