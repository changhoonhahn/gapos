\section{Results} \label{sec:results}
From the posteriors of galaxy properties inferred using
PROVABGS~(Section~\ref{sec:provabgs}), we derive the marginalized posteriors: 
$p(M_* \given {\bfi X_i})$, the marginalized 1D posterior of $M_*$ from
observed spectrophotometry ${\bfi X_i}$ of galaxy $i$.
Using these posteriors, we can estimate the probabilistic SMF (pSMF) of BGS
galaxies using population inference in a hierarchical Bayesian 
framework~\citep[\emph{e.g.}][]{hogg2010, foreman-mackey2014, baronchelli2020}.
In other words, we can infer $p(\phi\given\{{\bfi X_i}\})$, the probability
distribution of $\phi$ given the full DESI BGS observations, $\{{\bfi X_i}\}$. 
$\phi$ is the set of population hyperparameters that describe the pSMF,
$\Phi(M_*; \phi)$.
We again emphasize that this approach is statistically rigorous and correctly
propagates the uncertainties in our $M_*$ measurements to the pSMF. 

%We take this approach over the standard approach that use point estimates of
%$M_*$ because we can correctly propagate the uncertainties in our $M_*$
%measurements and more robustly estimate the $M_*$ distribution --- \emph{ie.}
%SMF. 
%\cite{malz2020} demonstrated in the context of inferring redshift
%distributions from individual photometric redshift measurements that using
%point estimates is statistically incorrect and can lead to biased redshift
%distributions. 
%We emphasize that \cite{malz2020} is an analogous analysis with a similar goal
%of measuring a 1D galaxy property distribution. 

In this work, we estimate the pSMF using a Gaussian Mixture
Model~\citep[GMM;][]{press1992, mclachlan2000}, which provides a highly flexible
description of the $M_*$ distribution: 
\begin{equation}
    \Phi(M_*; \phi) = \sum\limits_{j=1}^{k} \mathcal{N}(M_*; \phi_j).
\end{equation} 
$k$ represents the number of Gaussian components. 
$\phi_j$ represent the mean and standard deviation of the $j^{\rm th}$ Gaussian
component of the GMM. 
Previous works have used parametric functions (\emph{e.g.} Schechter function)
to describe the pSMF~\citep{leja2019a}. 
We opt for GMMs in order to produce a non-parametric measuerment of the pSMF.
In a subsequent work, Speranza~\etal~(in prep.), we will present the BGS pSMF
measured using a parametric model with continuous redshift evolution.  


To infer $p(\phi\given\{{\bfi X_i}\})$, we follow the same approach described
in \cite{hahn2022}:
\begin{align}\label{eq:popinf}
p(\phi \given \{{\bfi X_i}\}) 
    =&~\frac{p(\phi)~p( \{{\bfi X_i}\} \given \phi)}{p(\{{\bfi X_i}\})}\\
    =&~\frac{p(\phi)}{p(\{{\bfi X_i}\})}\int p(\{{\bfi X_i}\} \given \{\theta_i\})~p(\{\theta_i\} \given \phi)~{\rm d}\{\theta_i\}.\\
    =&~\frac{p(\phi)}{p(\{{\bfi X_i}\})}\prod\limits_{i=1}^N\int p({\bfi X_i} \given \theta_i)~p(\theta_i \given \phi)~{\rm d}\theta_i\\
    =&~\frac{p(\phi)}{p(\{{\bfi X_i}\})}\prod\limits_{i=1}^N\int \frac{p(\theta_i \given {\bfi X_i})~p({\bfi X_i})}{p(\theta_i)}~p(\theta_i \given \phi)~{\rm d}\theta_i\\
    =&~p(\phi)\prod\limits_{i=1}^N\int \frac{p(\theta_i \given {\bfi X_i})~p(\theta_i \given \phi)}{p(\theta_i)}~{\rm d}\theta_i. 
\end{align}
\noindent We can estimate the integral using $S_i$ Monte Carlo samples from
the individual posteriors $p(\theta_i \given {\bfi X_i})$: 
\begin{align}
    p(\phi \given \{{\bfi X_i}\}) \approx~p(\phi)\prod\limits_{i=1}^N\frac{1}{S_i}\sum\limits_{j=1}^{S_i}
    \frac{p(\theta_{i,j} \given \phi)}{p(\theta_{i,j})}.
\end{align} 

%BGS provides two samples: BGS Bright and Faint. 
%Galaxies in BGS Bright are selected based on a $r < 19.5$ flux limit, while
%the ones in BGS Faint are selected based on a fiber-magnitude and color limit
%and $r < 20.0175$ flux limit. 
Since the sample of BGS galaxies is not volume-limited and complete as a
function of $M_*$, we must account for the selection effect and incompleteness
when estimating the pSMF. 
To account for the selection effects of the BGS samples, we include weights
derived from $z^{\rm max}_i$, the maximum redshift that galaxy $i$ could have
and still be included in the BGS samples. 
We derive $z^{\rm max}_i$ for every galaxy using by redshifting the SED
predicted by the best-fit parameters. 
We then derive $V^{\rm max}_i$, the comoving volume out to $z^{\rm max}_i$, and
include a factor of $1/V^{\rm max}_i$ in the galaxy weight $w_i$. 

Next, we include correction weights for spectroscopic incompleteness driven by
fiber assignment and redshift failures. 
Incompletenss from fiber assignment is due to the fact that DESI is not able to
assign fibers to all galaxies included in the BGS target selection. 
Furthermore, due to the clustering of galaxies there is significant variation
in the assignment probability. 
Meanwhile, incompleteness from redshift failure is caused by the fact that we
do not successfully measure the redshift for every spectra and the redshift
failure rate depends significantly on the surface brightnesses of the galaxies
and the signal-to-noise ratio of the spectra. 
We describe how we derive  the incompleteness correction weights for fiber
assignment and redshift failures, $w_{i, {\rm FA}}$ and $w_{i, {\rm ZF}}$, in
Appendix~\ref{sec:spec_comp}. 
Each BGS galaxy is assigned a weight of 
$w_i = (w_{i, {\rm FA}}\times w_{i, {\rm ZF}})/V^{\rm max}_i$.

We modify Eq.~\ref{eq:popinf} to include galaxy weights, $w_i$: 
\begin{align}
p(\phi \given \{{\bfi X_i}\}) 
    \approx&~\frac{p(\phi)}{\prod\limits_{i=1}^N p({\bfi X_i})^{w_i}} 
    \prod\limits_{i=1}^N \left(\int p({\bfi X_i} \given \theta_i)~p(\theta_i \given \phi)~{\rm d}\theta_i \right)^{w_i} \\ 
    \approx&~\frac{p(\phi)}{\prod\limits_{i=1}^N p({\bfi X_i})^{w_i}} 
    \prod\limits_{i=1}^N \left( \sum\limits_{j=1}^{S_i}
    \frac{p(\theta_{i,j} \given \phi)}{p(\theta_{i,j})} \right)^{w_i} \\
    \approx&~\frac{p(\phi)}{\prod\limits_{i=1}^N p({\bfi X_i})^{w_i}} 
    \prod\limits_{i=1}^N \left( \sum\limits_{j=1}^{S_i}
    \frac{q_\phi(\theta_{i,j})}{p(\theta_{i,j})} \right)^{w_i}.
\end{align} 

In practice, we do not derive the full posterior 
$p(\phi \given \{{\bfi X_i}\})$. 
Instead we derive the maximum a posteriori (MAP) hyperparameter 
$\phi_{\rm MAP}$ that maximizes $p(\phi \given \{{\bfi X_i}\})$ or 
$\log p(\phi \given \{{\bfi X_i}\})$.
We expand, 
\begin{align}
\log p(\phi \given \{{\bfi X_i}\}) 
    \approx&~\log p(\phi) + % \sum\limits_{i=1}^N w_i \log w_i + 
    \sum\limits_{i=1}^N w_i \log \left(\sum\limits_{j=1}^{S_i} \frac{q_\phi(\theta_{i,j})}{p(\theta_{i,j})} \right).
\end{align} 
Since the first two terms are constant, we derive $\phi_{\rm MAP}$ by
maximizing 
\begin{equation}
    \max_\phi~~\sum\limits_{i=1}^N w_i \log \left(\sum\limits_{j=1}^{S_i} \frac{q_\phi(\theta_{i,j})}{p(\theta_{i,j})} \right).
\end{equation}
using the {\sc Adam} optimizer~\citep{kingma2017}.  
We derive $\phi_{\rm MAP}$ for BGS galaxies in redshift bins of width 
$\Delta z = 0.04$ starting from $z =0.01$ in order to examine the redshift
evolution of the SMF within BGS. 

%\approx&~\log p(\phi) - 
%\log \prod\limits_{i=1}^N p({\bfi X_i})^{w_i} + 
%\log \prod\limits_{i=1}^N \left(\int p({\bfi X_i} \given \theta_i)~p(\theta_i \given \phi)~{\rm d}\theta_i \right)^{w_i} \\
%\approx&~\log p(\phi) - 
%\sum\limits_{i=1}^N w_i \log p({\bfi X_i}) + 
%\sum\limits_{i=1}^N w_i \log \left(\int p({\bfi X_i} \given \theta_i)~p(\theta_i \given \phi)~{\rm d}\theta_i \right) \\
%\approx&~\log p(\phi) + 
%\sum\limits_{i=1}^N w_i \log \left(\frac{1}{w_i} \sum\limits_{j=1}^{S_i} w_{i,j} \frac{p(\theta_{i,j} \given \phi)}{p(\theta_{i,j}} \right) \\

%\subsection{Targeting Completeness} \label{sec:ts}
% https://desi.lbl.gov/trac/wiki/ClusteringWG/LSScat/SV3/version2.1/fulldat
% https://desi.lbl.gov/trac/wiki/ClusteringWG/LSScat/SV3/version2.1/fullran

\begin{figure}
\begin{center}
    \includegraphics[width=0.9\textwidth]{figs/psmf_bgs_any_comp.pdf} 
    \caption{
        The probabilistic SMF (pSMF) of BGS galaxies in the One-Percent Survey
        at $0.01 < z < 0.05$ (black line). 
        We represent uncertainties on the pSMF, estimated using a standard
        jackknife technique (Appendix~\ref{sec:jack}), in the shaded regions.
        The solid line represents the pSMF above the completeness limit 
        $M_* > M_{\rm lim} = 10^{8.975}M_\odot$ (Appendix~\ref{sec:mscomp}).
        In the left panel, we present the pSMFs of BGS Bright (blue) and
        Faint (orange) galaxies. 
        In the right panel, we include SMF measurements from previous
        spectroscopic surveys for comparison: SDSS~\citep{moustakas2013,
        bernardi2017} and GAMA~\citep{driver2022}. 
        Overall, the pSMF of BGS are in good agreement with SMF
        measurements from previous surveys.  
    }\label{fig:psmf}
\end{center}
\end{figure}

\begin{figure}
\begin{center}
    \includegraphics[width=0.5\textwidth]{figs/psmf_bgs_any_zevo.pdf} 
    \caption{
        The BGS pSMF over the redshift range $0.01 < z < 0.17$ in bins of
        $\Delta z = 0.04$. 
        The shaded regions represent the uncertainties on the pSMF, estimated
        using a standard jackknife technique.
        The solid lines represent the pSMF above the completeness limit 
        $M_* > M_{\rm lim}$ while the dashed lines represent the pSMF below
        the limit.
        There is no significant redshift evolution of the pSMF given the
        statistical uncertainties. 
        The main BGS survey will observe $>100\times$ more galaxies than the
        One-Precent Survey. 
    }\label{fig:psmfz}
\end{center}
\end{figure}

\subsection{The Probabilistic Stellar Mass Function} \label{sec:psmf}
We present the probabilistic SMF (pSMF) of $0.01 < z < 0.05$ BGS galaxies in the
One-Percent Survey in Figure~\ref{fig:psmf} (black line). 
The shaded regions represent the uncertainties of the pSMF from sample variance,
which we derive using a standard jackknife technique
(Appendix~\ref{sec:jack}) and are conservative estimates~\citep{norberg2009}. 
In the left panel, we also present the pSMFs of the BGS Bright (blue) and Faint
(orange) galaxies.  
BGS Bright galaxies are selected using a $r > 19.5$ magnitude limit. 
As a result, the BGS Bright sample is $M_*$ complete above $M_{\rm lim} >
10^{8.975}M_\odot$. 
We derive $M_{\rm lim}$ in Appendix~\ref{sec:mscomp} and mark the pSMF above
the completeness limit in solid and below the limit in dashed. 
Meanwhile, the BGS Faint sample is selected using a surface brightness and
color selection.   
It includes fainter galaxies, $19.5 < r < 20.175$, with overall lower $M_*$
than the BGS Bright sample. 

We also include the SMF estimated using the standard approach (black dotted). 
This SMF is derived using the best-fit $M_*$ point estimates with the same
galaxy weights, $w_i$. 
At intermediate $M_*$ range, $10^9 < M_* < 10^{11}M_\odot$, we find good
agrement with the pSMF. 
However, the standard approach significantly underestimates the SMF outside
this $M_*$ range. 
These discrepancies is due to the fact that point estimates of $M_*$ ignore the
uncertainties which contribute significantly at the most and least massive ends
of the SMF.
The discpreancies are present in all other redshift bins and underscore the
importance of correctly propagating the $M_*$ uncertainties. 

In the right panel, we compare the BGS pSMF to SMF measurements from previous 
spectroscopic surveys: SDSS~\citep{moustakas2013, bernardi2017} (black circle
and square) and GAMA~\citep{driver2022} (black triangle).
%For the \cite{driver2022} SMF, we include a 0.0807 dex correction that the authors recommend to re-normalize the SMF and correction to $z=0$. 
We note that there is significant variance in SMF measurements in the
literature, especially at the high $M_*$ end. 
This is partly due to the different modeling methodologies used to derive
$M_*$, which can contribute >0.1 dex discrepancies~\citep{pacifici2023}. 
Furthermore, there are also discrepancies due to photometric corrections
applied to SDSS photometry, assumptions on the stellar populations, and
dust~\citep{bernardi2017}.
In a subsequent work, we will present a detailed comparison of BGS $M_*$
measurements using different methods. 
Overall, we find good agreement with previous SMF measurements, especially in
the intermediate $M_*$ range where we precisely infer the pSMF.  

In Figure~\ref{fig:psmfz}, we present the redshift evolution of the pSMF over 
$0.01 < z < 0.17$ in redshift bins of $\Delta z = 0.04$. 
The shaded region represent the jackknife uncertainties for the pSMF.
The solid line represents the pSMF above $M_{\rm lim}$ while the dashed lines
represent the pSMF below the limit. 
We only include 4 redshift bins, since $M_{\rm lim} > 10^{10.5}M_\odot$ for 
$z > 0.17$ (Table~\ref{tab:mscomp}).
The pSMFs in Figure~\ref{fig:psmfz} do not reveal a significant redshift
dependence given their uncertainties. 
We note that the large uncertainties for the $0.01 < z < 0.05$ pSMF is driven
by large-scale structure at RA $\sim 195$ deg, Dec $\sim 28$ deg, and 
$z\sim0.244$. 
The main BGS survey will observe $>100\times$ more BGS galaxies than the
One-Percent Survey and enable pSMF measurements with unprecedented precision. 

\begin{figure}
\begin{center}
    \includegraphics[width=0.5\textwidth]{figs/sfq.pdf} 
    \caption{
        The $M_*-{\rm sSFR}$ distribution of BGS galaxies at $z < 0.2$. 
        sSFR is derived using $\overline{\rm SFR}$, average SFR over the last 1
        Gyr, inferred using PROVABGS. 
        The $M_*-{\rm sSFR}$ distribution is bimodal with star-forming galaxies
        lying on the star-forming sequence.
        We classify galaxies with ${\rm sSFR} > 10^{-11.2}\,{\rm yr}^{-1}$ as
        star-forming galaxies and ${\rm sSFR} < 10^{-11.2}\,{\rm yr}^{-1}$ as
        quiescent. 
    }\label{fig:sfq}
\end{center}
\end{figure}


\begin{figure}
\begin{center}
    \includegraphics[width=0.9\textwidth]{figs/psmf_bgs_bright_sfq.pdf} 
    \caption{
        The pSMF of star-forming (left) and quiescent (right) BGS Bright
        galaxies over $0.01 < z < 0.17$ in bins of $\Delta z = 0.04$. 
        Star-forming and quiescent galaxies are classified using an empirically
        determined ${\rm sSFR} = 10^{-11.2}{\rm yr}^{-1}$ cut. 
        We represent the uncertainties for the pSMF in the shaded regions and 
        the pSMFs above/below the $M_*$ completeness limits in solid/dashed
        lines.
        The pSMFs suggest a decline in massive, $M_* > 10^{11}M_\odot$ 
        star-forming galaxies and an increase in the quiescent galaxy population
        at $M_* < 10^{11}M_\odot$ at lower redshifts.
    }\label{fig:sfqsmf}
\end{center}
\end{figure}

\subsection{Star-Forming and Quiescent Galaxies in the BGS} \label{sec:sfq}
In addition to the pSMF of the full galaxy population, we can also examine the
pSMF of the star-forming and quiescent subpopulations using 
$\overline{\rm SFR}$, average SFR over the last 1 Gyr, inferred using PROVABGS. 
In Figure~\ref{fig:sfq}, we present the distribution of $M_*$ versus average
specific SFR, $\overline{\rm sSFR} = \overline{\rm SFR}/M_*$, for BGS Bright
(blue) and Faint (orange) galaxies at $z < 0.2$. 
The $M_*-{\rm sSFR}$ distribution of the BGS galaxies reveal a clear
bimodality with star-forming galaxies lying on the SFS and quiescent galaxies
lying $\gtrsim$1 dex below the sequence. 
Figure~\ref{fig:sfq} also confirms that BGS Faint galaxies have overall lower
$M_*$ than BGS Bright galaxies and are primarily star-forming galaxies. 
This is due to the fact that the $(z - W1)-1.2(g-r)+1.2$ color used to select
BGS Faint galaxies is a proxy for H$\alpha$ and H$\beta$ emission lines.  

To further examine the star-forming and quiescent galaxy populations, we
classify BGS Bright galaxies as star-forming or quiescent using a 
$\overline{\rm sSFR} = 10^{-11.2}\,{\rm yr}^{-1}$ cut. 
We determine this cut empirically based roughly on the sSFR of the ``green
valley'' between the SFS and the quiescent mode. 
We opt for a $\overline{\rm sSFR}$ cut rather than more sophisticated methods
in the literature~\citep[\emph{e.g.}][]{hahn2019, donnari2019} for simplicity. 
In Figure~\ref{fig:sfqsmf}, we present the pSMF of star-forming and quiescent
BGS Bright galaxies at $0.01 < z < 0.17$ in bins of $\Delta z = 0.04$.
The shaded regions represent the jackknife uncertainties for the pSMF. 
The solid lines represent the pSMFs above the completeness limit while the
dashed lines represent the pSMFs below the limit. 
The pSMF of quiescent galaxies suggest an increase in the number of galaxies
below $M_* < 10^{11}M_\odot$.
Meanwhile, the pSMF of star-forming galaxies shows a possible decline at the
massive end over $0.01 < z < 0.17$. 

\begin{figure}
\begin{center}
    \includegraphics[width=0.45\textwidth]{figs/qf_bgs_bright.pdf} 
    \caption{
        The quiescent fraction of BGS Bright galaxies over $0.01 < z < 0.17$ in
        bins of $\Delta z =0.04$.
        We present the uncertainties in the shaded region and only include  
        the quiescent fraction above the $M_*$ completeness limit. 
        The quiescent fractions increase with $M_*$ at all redshifts. 
        Furthermore, the quiescent fractions suggest an overall increase in the
        quiescent population with lower redshift.
    }\label{fig:qf}
\end{center}
\end{figure}

Next, we present the fraction of quiescent galaxies in BGS Bright as a function
of $M_*$ over $0.01 < z < 0.17$ in Figure~\ref{fig:qf}.
The quiescent fraction is derived by taking the ratio of the pSMFs of quiescent
galaxies over all galaxies and measured for each $\Delta z =0.04$ bin.
The shaded region represent the uncertainties derived from propagating the
jackknife uncertainties of the pSMFs. 
We focus on the quiescent fraction of BGS Bright galaxies above the $M_*$
completeness limit: $M_* > M_{\rm lim}$. 
At each redshift bin, the quiescent fraction increases with $M_*$ to $\sim$1 at
$M_*\sim10^{11.5}M_\odot$. 
The quiescent fraction also suggests an increase in the quiescent population
with redshift. 
Although the significant statistical uncertainties obfuscate a clear trend, 
the quiescent fraction evolution is in good qualitative agreement with previous
works~\citep[\emph{e.g.}][]{baldry2006, iovino2010, peng2010, hahn2015}.
Upcoming observations from the DESI main survey will increase the number of BGS
galaxies by >100$\times$ and enable precise comparisons of the quiescent
fraction measurements. 
